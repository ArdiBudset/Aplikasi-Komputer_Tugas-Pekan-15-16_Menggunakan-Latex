\documentclass{article}

\usepackage{eumat}

\begin{document}
\begin{eulernotebook}
\eulersubheading{}
\begin{eulercomment}
Nama : Ardi Budi Setiawan\\
NIM  : 22305141017\\
Kelas: Matematika-B


\begin{eulercomment}
\eulerheading{Visualisasi dan Perhitungan Geometri dengan EMT}
\begin{eulercomment}
Euler menyediakan beberapa fungsi untuk melakukan visualisasi dan
perhitungan geometri, baik secara numerik maupun analitik (seperti
biasanya tentunya, menggunakan Maxima). Fungsi-fungsi untuk
visualisasi dan perhitungan geometeri tersebut disimpan di dalam file
program "geometry.e", sehingga file tersebut harus dipanggil sebelum
menggunakan fungsi-fungsi atau perintah-perintah untuk geometri.
\end{eulercomment}
\begin{eulerprompt}
>load geometry
\end{eulerprompt}
\begin{euleroutput}
  Numerical and symbolic geometry.
\end{euleroutput}
\eulersubheading{Fungsi-fungsi Geometri}
\begin{eulercomment}
Fungsi-fungsi untuk Menggambar Objek Geometri:

\end{eulercomment}
\begin{eulerttcomment}
  defaultd:=textheight()*1.5: nilai asli untuk parameter d
  setPlotrange(x1,x2,y1,y2): menentukan rentang x dan y pada bidang
\end{eulerttcomment}
\begin{eulercomment}
koordinat\\
\end{eulercomment}
\begin{eulerttcomment}
  setPlotRange(r): pusat bidang koordinat (0,0) dan batas-batas
\end{eulerttcomment}
\begin{eulercomment}
sumbu-x dan y adalah -r sd r\\
\end{eulercomment}
\begin{eulerttcomment}
  plotPoint (P, "P"): menggambar titik P dan diberi label "P"
  plotSegment (A,B, "AB", d): menggambar ruas garis AB, diberi label
\end{eulerttcomment}
\begin{eulercomment}
"AB" sejauh d\\
\end{eulercomment}
\begin{eulerttcomment}
  plotLine (g, "g", d): menggambar garis g diberi label "g" sejauh d
  plotCircle (c,"c",v,d): Menggambar lingkaran c dan diberi label "c"
  plotLabel (label, P, V, d): menuliskan label pada posisi P
\end{eulerttcomment}
\begin{eulercomment}

Fungsi-fungsi Geometri Analitik (numerik maupun simbolik):

\end{eulercomment}
\begin{eulerttcomment}
  turn(v, phi): memutar vektor v sejauh phi
  turnLeft(v):   memutar vektor v ke kiri
  turnRight(v):  memutar vektor v ke kanan
  normalize(v): normal vektor v
  crossProduct(v, w): hasil kali silang vektorv dan w.
  lineThrough(A, B): garis melalui A dan B, hasilnya [a,b,c] sdh.
\end{eulerttcomment}
\begin{eulercomment}
ax+by=c.\\
\end{eulercomment}
\begin{eulerttcomment}
  lineWithDirection(A,v): garis melalui A searah vektor v
  getLineDirection(g): vektor arah (gradien) garis g
  getNormal(g): vektor normal (tegak lurus) garis g
  getPointOnLine(g):  titik pada garis g
  perpendicular(A, g):  garis melalui A tegak lurus garis g
  parallel (A, g):  garis melalui A sejajar garis g
  lineIntersection(g, h):  titik potong garis g dan h
  projectToLine(A, g):   proyeksi titik A pada garis g
  distance(A, B):  jarak titik A dan B
  distanceSquared(A, B):  kuadrat jarak A dan B
  quadrance(A, B): kuadrat jarak A dan B
  areaTriangle(A, B, C):  luas segitiga ABC
  computeAngle(A, B, C):   besar sudut <ABC
  angleBisector(A, B, C): garis bagi sudut <ABC
  circleWithCenter (A, r): lingkaran dengan pusat A dan jari-jari r
  getCircleCenter(c):  pusat lingkaran c
  getCircleRadius(c):  jari-jari lingkaran c
  circleThrough(A,B,C):  lingkaran melalui A, B, C
  middlePerpendicular(A, B): titik tengah AB
  lineCircleIntersections(g, c): titik potong garis g dan lingkran c
  circleCircleIntersections (c1, c2):  titik potong lingkaran c1 dan
\end{eulerttcomment}
\begin{eulercomment}
c2\\
\end{eulercomment}
\begin{eulerttcomment}
  planeThrough(A, B, C):  bidang melalui titik A, B, C
\end{eulerttcomment}
\begin{eulercomment}

Fungsi-fungsi Khusus Untuk Geometri Simbolik:

\end{eulercomment}
\begin{eulerttcomment}
  getLineEquation (g,x,y): persamaan garis g dinyatakan dalam x dan y
  getHesseForm (g,x,y,A): bentuk Hesse garis g dinyatakan dalam x dan
\end{eulerttcomment}
\begin{eulercomment}
y dengan titik A pada\\
\end{eulercomment}
\begin{eulerttcomment}
  sisi positif (kanan/atas) garis
  quad(A,B): kuadrat jarak AB
  spread(a,b,c): Spread segitiga dengan panjang sisi-sisi a,b,c, yakni
\end{eulerttcomment}
\begin{eulercomment}
sin(alpha)\textasciicircum{}2 dengan\\
\end{eulercomment}
\begin{eulerttcomment}
  alpha sudut yang menghadap sisi a.
  crosslaw(a,b,c,sa): persamaan 3 quads dan 1 spread pada segitiga
\end{eulerttcomment}
\begin{eulercomment}
dengan panjang sisi a, b, c.\\
\end{eulercomment}
\begin{eulerttcomment}
  triplespread(sa,sb,sc): persamaan 3 spread sa,sb,sc yang memebntuk
\end{eulerttcomment}
\begin{eulercomment}
suatu segitiga\\
\end{eulercomment}
\begin{eulerttcomment}
  doublespread(sa): Spread sudut rangkap Spread 2*phi, dengan
\end{eulerttcomment}
\begin{eulercomment}
sa=sin(phi)\textasciicircum{}2 spread a.

\end{eulercomment}
\eulersubheading{Contoh 1: Luas, Lingkaran Luar, Lingkaran Dalam Segitiga}
\begin{eulercomment}
Untuk menggambar objek-objek geometri, langkah pertama adalah
menentukan rentang sumbu-sumbu koordinat. Semua objek geometri akan
digambar pada satu bidang koordinat, sampai didefinisikan bidang
koordinat yang baru.
\end{eulercomment}
\begin{eulerprompt}
>setPlotRange(-0.5,2.5,-0.5,2.5); // mendefinisikan bidang koordinat baru 
\end{eulerprompt}
\begin{eulercomment}
Sekarang atur tiga poin dan plot.
\end{eulercomment}
\begin{eulerprompt}
>A=[1,0]; plotPoint(A,"A"); // definisi dan gambar tiga titik
>B=[0,1]; plotPoint(B,"B");
>C=[2,2]; plotPoint(C,"C");
\end{eulerprompt}
\begin{eulercomment}
Lalu tiga segmen.
\end{eulercomment}
\begin{eulerprompt}
>plotSegment(A,B,"c"); // c=AB
>plotSegment(B,C,"a"); // a=BC
>plotSegment(A,C,"b"); // b=AC
\end{eulerprompt}
\begin{eulercomment}
Fungsi geometri meliputi fungsi untuk membuat garis dan lingkaran.
Format untuk garis adalah [a, b, c], yang merepresentasikan garis
dengan persamaan ax + by = c.
\end{eulercomment}
\begin{eulerprompt}
>lineThrough(B,C) // garis yang melalui B dan C
\end{eulerprompt}
\begin{euleroutput}
  [-1,  2,  2]
\end{euleroutput}
\begin{eulercomment}
Hitung garis tegak lurus melalui A pada BC.
\end{eulercomment}
\begin{eulerprompt}
>h=perpendicular(A,lineThrough(B,C)); // garis h tegak lurus BC melalui A
\end{eulerprompt}
\begin{eulercomment}
Dan perpotongannya dengan BC.
\end{eulercomment}
\begin{eulerprompt}
>D=lineIntersection(h,lineThrough(B,C)); // D adalah titik potong h dan BC
\end{eulerprompt}
\begin{eulercomment}
Plot itu.
\end{eulercomment}
\begin{eulerprompt}
>plotPoint(D,value=1); // koordinat D ditampilkan
>aspect(1); plotSegment(A,D): // tampilkan semua gambar hasil plot...()
\end{eulerprompt}
\eulerimg{27}{images/22305141017_Ardi Budi S_Geometri-001.png}
\begin{eulercomment}
Hitung luas ABC:

\end{eulercomment}
\begin{eulerformula}
\[
L_{\triangle ABC}= \frac{1}{2}AD.BC.
\]
\end{eulerformula}
\begin{eulerprompt}
>norm(A-D)*norm(B-C)/2 // AD=norm(A-D), BC=norm(B-C)
\end{eulerprompt}
\begin{euleroutput}
  1.5
\end{euleroutput}
\begin{eulercomment}
Cara lain menghitung rumus determinan.
\end{eulercomment}
\begin{eulerprompt}
>areaTriangle(A,B,C) // hitung luas segitiga langusng dengan fungsi
\end{eulerprompt}
\begin{euleroutput}
  1.5
\end{euleroutput}
\begin{eulercomment}
Cara lain menghitung luas segitigas ABC:
\end{eulercomment}
\begin{eulerprompt}
>distance(A,D)*distance(B,C)/2
\end{eulerprompt}
\begin{euleroutput}
  1.5
\end{euleroutput}
\begin{eulercomment}
Sudut di C
\end{eulercomment}
\begin{eulerprompt}
>degprint(computeAngle(B,C,A))
\end{eulerprompt}
\begin{euleroutput}
  36°52'11.63''
\end{euleroutput}
\begin{eulercomment}
Sekarang lingkaran sirkit segitiga.
\end{eulercomment}
\begin{eulerprompt}
>c=circleThrough(A,B,C); // lingkaran luar segitiga ABC
>R=getCircleRadius(c); // jari2 lingkaran luar 
>O=getCircleCenter(c); // titik pusat lingkaran c 
>plotPoint(O,"O"); // gambar titik "O"
>plotCircle(c,"Lingkaran luar segitiga ABC"):
\end{eulerprompt}
\eulerimg{27}{images/22305141017_Ardi Budi S_Geometri-003.png}
\begin{eulercomment}
Tampilkan koordinat titik pusat dan jari-jari lingkaran luar.
\end{eulercomment}
\begin{eulerprompt}
>O, R
\end{eulerprompt}
\begin{euleroutput}
  [1.16667,  1.16667]
  1.17851130198
\end{euleroutput}
\begin{eulercomment}
Sekarang akan digambar lingkaran dalam segitiga ABC. Titik pusat lingkaran dalam adalah
titik potong garis-garis bagi sudut.
\end{eulercomment}
\begin{eulerprompt}
>l=angleBisector(A,C,B); // garis bagi <ACB
>g=angleBisector(C,A,B); // garis bagi <CAB
>P=lineIntersection(l,g) // titik potong kedua garis bagi sudut
\end{eulerprompt}
\begin{euleroutput}
  [0.86038,  0.86038]
\end{euleroutput}
\begin{eulercomment}
Tambahkan semua ke plot.
\end{eulercomment}
\begin{eulerprompt}
>color(5); plotLine(l); plotLine(g); color(1); // gambar kedua garis bagi sudut
>plotPoint(P,"P"); // gambar titik potongnya
>r=norm(P-projectToLine(P,lineThrough(A,B))) // jari-jari lingkaran dalam
\end{eulerprompt}
\begin{euleroutput}
  0.509653732104
\end{euleroutput}
\begin{eulerprompt}
>plotCircle(circleWithCenter(P,r),"Lingkaran dalam segitiga ABC"): // gambar lingkaran dalam
\end{eulerprompt}
\eulerimg{27}{images/22305141017_Ardi Budi S_Geometri-004.png}
\eulersubheading{Latihan}
\begin{eulercomment}
1. Tentukan ketiga titik singgung lingkaran dalam dengan sisi-sisi
segitiga ABC.\\
2. Gambar segitiga dengan titik-titik sudut ketiga titik singgung
tersebut. Merupakan segitiga apakah itu?\\
3. Hitung luas segitiga tersebut.\\
4. Tunjukkan bahwa garis bagi sudut yang ke tiga juga melalui titik
pusat lingkaran dalam.\\
5. Gambar jari-jari lingkaran dalam.\\
6. Hitung luas lingkaran luar dan luas lingkaran dalam segitiga ABC.
Adakah hubungan antara luas kedua lingkaran tersebut dengan luas
segitiga ABC?


Jawab:\\
\end{eulercomment}
\eulersubheading{}
\begin{eulercomment}
1. Tentukan ketiga titik singgung lingkaran dalam dengan sisi-sisi
segitiga ABC.\\
Titik singgung garis BC dengan lingkaran dalam.
\end{eulercomment}
\begin{eulerprompt}
>TBC=lineThrough(B,C)
\end{eulerprompt}
\begin{euleroutput}
  [-1,  2,  2]
\end{euleroutput}
\begin{eulerprompt}
>m=circleWithCenter(P,r)
\end{eulerprompt}
\begin{euleroutput}
  [0.86038,  0.86038,  0.509654]
\end{euleroutput}
\begin{eulercomment}
Dimisalkan S
\end{eulercomment}
\begin{eulerprompt}
>S=lineCircleIntersections(TBC,m)
\end{eulerprompt}
\begin{euleroutput}
  [0.632456,  1.31623]
\end{euleroutput}
\begin{eulercomment}
Titik singgung garis AC dengan lingkaran dalam.
\end{eulercomment}
\begin{eulerprompt}
>TAC=lineThrough(A,C)
\end{eulerprompt}
\begin{euleroutput}
  [-2,  1,  -2]
\end{euleroutput}
\begin{eulercomment}
Dimisalkan Q
\end{eulercomment}
\begin{eulerprompt}
>Q=lineCircleIntersections(TAC,m)
\end{eulerprompt}
\begin{euleroutput}
  [1.31623,  0.632456]
\end{euleroutput}
\begin{eulercomment}
Titik singgung garis AB dengan lingkaran dalam.
\end{eulercomment}
\begin{eulerprompt}
>TAB=lineThrough(A,B)
\end{eulerprompt}
\begin{euleroutput}
  [-1,  -1,  -1]
\end{euleroutput}
\begin{eulercomment}
Dimisalkan L
\end{eulercomment}
\begin{eulerprompt}
>L=lineCircleIntersections(TAB,m)
\end{eulerprompt}
\begin{euleroutput}
  [0.5,  0.5]
\end{euleroutput}
\begin{eulercomment}
Sehingga, titik singgung lingkaran dalam dengan sisi-sisi segitiga
yang diperoleh adalah (0.632456, 1.31623), (1.31632, 0.632456), (0.5,
0.5).

\end{eulercomment}
\eulersubheading{}
\begin{eulercomment}
2. Gambar segitiga dengan titik-titik sudut ketiga titik singgung
tersebut.
\end{eulercomment}
\begin{eulerprompt}
>plotPoint(S);
>plotPoint(Q);
>plotPoint(L);
>plotSegment(S,Q,"a");
>plotSegment(S,L,"b");
>plotSegment(L,Q,"c"):
\end{eulerprompt}
\eulerimg{27}{images/22305141017_Ardi Budi S_Geometri-005.png}
\begin{eulercomment}
\end{eulercomment}
\eulersubheading{}
\begin{eulercomment}
3. Hitung luas segitiga tersebut.
\end{eulercomment}
\begin{eulerprompt}
>areaTriangle(S,L,Q)
\end{eulerprompt}
\begin{euleroutput}
  0.324341649025
\end{euleroutput}
\begin{eulercomment}
Diperoleh luas segitiga diatas, yaitu 0,324341649025
\end{eulercomment}
\begin{eulercomment}

\end{eulercomment}
\eulersubheading{}
\begin{eulercomment}
4. Tunjukkan bahwa garis bagi sudut yang ke tiga juga melalui titik
pusat lingkaran dalam.
\end{eulercomment}
\begin{eulerprompt}
> P, r
\end{eulerprompt}
\begin{euleroutput}
  [0.86038,  0.86038]
  0.509653732104
\end{euleroutput}
\begin{eulerprompt}
>k=angleBisector(A,B,C)
\end{eulerprompt}
\begin{euleroutput}
  [-0.264911,  -1.63246,  -1.63246]
\end{euleroutput}
\begin{eulerprompt}
>color(2); plotLine(k):
\end{eulerprompt}
\eulerimg{27}{images/22305141017_Ardi Budi S_Geometri-006.png}
\begin{eulercomment}
\end{eulercomment}
\eulersubheading{}
\begin{eulercomment}
5. Gambar jari-jari lingkaran dalam.
\end{eulercomment}
\begin{eulerprompt}
>plotSegment(P,L,"r"):
\end{eulerprompt}
\eulerimg{27}{images/22305141017_Ardi Budi S_Geometri-007.png}
\begin{eulercomment}
\end{eulercomment}
\eulersubheading{}
\begin{eulercomment}
6. Hitung luas lingkaran luar dan luas lingkaran dalam segitiga ABC.
Adakah hubungan antara luas kedua lingkaran tersebut dengan luas
segitiga ABC?

Luas lingkaran luar segitiga ABC
\end{eulercomment}
\begin{eulerprompt}
>r1 = distance(O,C);
>pi*(r1)^2
\end{eulerprompt}
\begin{euleroutput}
  4.36332312999
\end{euleroutput}
\begin{eulercomment}
Luas Lingkaran dalam segitiga ABC
\end{eulercomment}
\begin{eulerprompt}
>r2= distance(P,S);
>pi*(r2)^2
\end{eulerprompt}
\begin{euleroutput}
  0.81601903655
\end{euleroutput}
\eulerheading{Contoh 2: Geometri Smbolik}
\begin{eulercomment}
Kita dapat menghitung geometri tepat dan simbolis menggunakan Maxima.

Geometri file.e menyediakan fungsi yang sama (dan lebih banyak lagi)
di Maxima. Namun, sekarang kita dapat menggunakan perhitungan
simbolik.
\end{eulercomment}
\begin{eulerprompt}
>A &= [1,0]; B &= [0,1]; C &= [2,2]; // menentukan tiga titik A, B, C
\end{eulerprompt}
\begin{eulercomment}
Fungsi garis dan lingkaran bekerja seperti fungsi Euler, tetapi
menyediakan penghitungan simbolik.
\end{eulercomment}
\begin{eulerprompt}
>c &= lineThrough(B,C) // c=BC
\end{eulerprompt}
\begin{euleroutput}
  
                               [- 1, 2, 2]
  
\end{euleroutput}
\begin{eulercomment}
Kita bisa mendapatkan persamaan untuk sebuah garis dengan mudah.
\end{eulercomment}
\begin{eulerprompt}
>$getLineEquation(c,x,y), $solve(%,y) | expand // persamaan garis c
\end{eulerprompt}
\begin{eulerformula}
\[
\left[ y=\frac{x}{2}+1 \right] 
\]
\end{eulerformula}
\eulerimg{1}{images/22305141017_Ardi Budi S_Geometri-009-large.png}
\begin{eulerprompt}
>$getLineEquation(lineThrough([x1,y1],[x2,y2]),x,y), $solve(%,y) // persamaan garis melalui(x1, y1) dan (x2, y2)
\end{eulerprompt}
\begin{eulerformula}
\[
\left[ y=\frac{-\left({\it x_1}-x\right)\,{\it y_2}-\left(x-  {\it x_2}\right)\,{\it y_1}}{{\it x_2}-{\it x_1}} \right] 
\]
\end{eulerformula}
\eulerimg{1}{images/22305141017_Ardi Budi S_Geometri-011-large.png}
\begin{eulerprompt}
>$getLineEquation(lineThrough(A,[x1,y1]),x,y) // persamaan garis melalui A dan (x1, y1)
\end{eulerprompt}
\begin{eulerformula}
\[
\left({\it x_1}-1\right)\,y-x\,{\it y_1}=-{\it y_1}
\]
\end{eulerformula}
\begin{eulerprompt}
>h &= perpendicular(A,lineThrough(B,C)) // h melalui A tegak lurus BC
\end{eulerprompt}
\begin{euleroutput}
  
                                [2, 1, 2]
  
\end{euleroutput}
\begin{eulerprompt}
>Q &= lineIntersection(c,h) // Q titik potong garis c=BC dan h
\end{eulerprompt}
\begin{euleroutput}
  
                                   2  6
                                  [-, -]
                                   5  5
  
\end{euleroutput}
\begin{eulerprompt}
>$projectToLine(A,lineThrough(B,C)) // proyeksi A pada BC
\end{eulerprompt}
\begin{eulerformula}
\[
\left[ \frac{2}{5} , \frac{6}{5} \right] 
\]
\end{eulerformula}
\begin{eulerprompt}
>$distance(A,Q) // jarak AQ
\end{eulerprompt}
\begin{eulerformula}
\[
\frac{3}{\sqrt{5}}
\]
\end{eulerformula}
\begin{eulerprompt}
>cc &= circleThrough(A,B,C); $cc // (titik pusat dan jari-jari) lingkaran melalui A, B, C
\end{eulerprompt}
\begin{eulerformula}
\[
\left[ \frac{7}{6} , \frac{7}{6} , \frac{5}{3\,\sqrt{2}} \right] 
\]
\end{eulerformula}
\begin{eulerprompt}
>r&=getCircleRadius(cc); $r , $float(r) // tampilkan nilai jari-jari
\end{eulerprompt}
\begin{eulerformula}
\[
1.178511301977579
\]
\end{eulerformula}
\eulerimg{0}{images/22305141017_Ardi Budi S_Geometri-017-large.png}
\begin{eulerprompt}
>$computeAngle(A,C,B) // nilai <ACB
\end{eulerprompt}
\begin{eulerformula}
\[
\arccos \left(\frac{4}{5}\right)
\]
\end{eulerformula}
\begin{eulerprompt}
>$solve(getLineEquation(angleBisector(A,C,B),x,y),y)[1] // persamaan garis bagi <ACB
\end{eulerprompt}
\begin{eulerformula}
\[
y=x
\]
\end{eulerformula}
\begin{eulerprompt}
>P &= lineIntersection(angleBisector(A,C,B),angleBisector(C,B,A)); $P // titik potong 2 garis bagi sudut
\end{eulerprompt}
\begin{eulerformula}
\[
\left[ \frac{\sqrt{2}\,\sqrt{5}+2}{6} , \frac{\sqrt{2}\,\sqrt{5}+2  }{6} \right] 
\]
\end{eulerformula}
\begin{eulerprompt}
>P() // hasilnya sama dengan perhitungan sebelumnya
\end{eulerprompt}
\begin{euleroutput}
  [0.86038,  0.86038]
\end{euleroutput}
\eulersubheading{Garis dan Lingkaran yang Berpotongan}
\begin{eulercomment}
Tentu saja, kita juga bisa memotong garis dengan lingkaran, dan
lingkaran dengan lingkaran.
\end{eulercomment}
\begin{eulerprompt}
>A &:= [1,0]; c=circleWithCenter(A,4);
>B &:= [1,2]; C &:= [2,1]; l=lineThrough(B,C);
>setPlotRange(5); plotCircle(c); plotLine(l);
\end{eulerprompt}
\begin{eulercomment}
Perpotongan garis dengan lingkaran mengembalikan dua titik dan jumlah
titik perpotongan.
\end{eulercomment}
\begin{eulerprompt}
>\{P1,P2,f\}=lineCircleIntersections(l,c);
>P1, P2, f
\end{eulerprompt}
\begin{euleroutput}
  [4.64575,  -1.64575]
  [-0.645751,  3.64575]
  2
\end{euleroutput}
\begin{eulerprompt}
>plotPoint(P1); plotPoint(P2):
\end{eulerprompt}
\eulerimg{27}{images/22305141017_Ardi Budi S_Geometri-021.png}
\begin{eulercomment}
Hal yang sama di Maxima.
\end{eulercomment}
\begin{eulerprompt}
>c &= circleWithCenter(A,4) // lingkaran dengan pusat A jari-jari 4
\end{eulerprompt}
\begin{euleroutput}
  
                                [1, 0, 4]
  
\end{euleroutput}
\begin{eulerprompt}
>l &= lineThrough(B,C) // garis l melalui B dan C
\end{eulerprompt}
\begin{euleroutput}
  
                                [1, 1, 3]
  
\end{euleroutput}
\begin{eulerprompt}
>$lineCircleIntersections(l,c) | radcan, // titik potong lingkaran c dan garis l
\end{eulerprompt}
\begin{eulerformula}
\[
\left[ \left[ \sqrt{7}+2 , 1-\sqrt{7} \right]  , \left[ 2-\sqrt{7}   , \sqrt{7}+1 \right]  \right] 
\]
\end{eulerformula}
\begin{eulercomment}
Akan ditunjukkan bahwa sudut-sudut yang menghadap bsuusr yang sama adalah sama besar.
\end{eulercomment}
\begin{eulerprompt}
>C=A+normalize([-2,-3])*4; plotPoint(C); plotSegment(P1,C); plotSegment(P2,C);
>degprint(computeAngle(P1,C,P2))
\end{eulerprompt}
\begin{euleroutput}
  69°17'42.68''
\end{euleroutput}
\begin{eulerprompt}
>C=A+normalize([-4,-3])*4; plotPoint(C); plotSegment(P1,C); plotSegment(P2,C);
>degprint(computeAngle(P1,C,P2))
\end{eulerprompt}
\begin{euleroutput}
  69°17'42.68''
\end{euleroutput}
\begin{eulerprompt}
>insimg;
\end{eulerprompt}
\eulerimg{27}{images/22305141017_Ardi Budi S_Geometri-023.png}
\eulersubheading{Garis Sumbu}
\begin{eulercomment}
Berikut adalah langkah-langkah menggambar garis sumbu ruas garis AB:

1. Gambar lingkaran dengan pusat A melalui B.\\
2. Gambar lingkaran dengan pusat B melalui A.\\
3. Tarik garis melallui kedua titik potong kedua lingkaran tersebut. Garis ini merupakan
garis sumbu (melalui titik tengah dan tegak lurus) AB.
\end{eulercomment}
\begin{eulerprompt}
>A=[2,2]; B=[-1,-2];
>c1=circleWithCenter(A,distance(A,B));
>c2=circleWithCenter(B,distance(A,B));
>\{P1,P2,f\}=circleCircleIntersections(c1,c2);
>l=lineThrough(P1,P2);
>setPlotRange(5); plotCircle(c1); plotCircle(c2);
>plotPoint(A); plotPoint(B); plotSegment(A,B); plotLine(l):
\end{eulerprompt}
\eulerimg{27}{images/22305141017_Ardi Budi S_Geometri-024.png}
\begin{eulercomment}
Selanjutnya, kita melakukan hal yang sama di Maxima dengan koordinat
umum.
\end{eulercomment}
\begin{eulerprompt}
>A &= [a1,a2]; B &= [b1,b2];
>c1 &= circleWithCenter(A,distance(A,B));
>c2 &= circleWithCenter(B,distance(A,B));
>P &= circleCircleIntersections(c1,c2); P1 &= P[1]; P2 &= P[2];
\end{eulerprompt}
\begin{eulercomment}
Persamaan untuk persimpangan cukup terlibat. Tapi kita bisa
menyederhanakan, jika kita menyelesaikan y.
\end{eulercomment}
\begin{eulerprompt}
>g &= getLineEquation(lineThrough(P1,P2),x,y);
>$solve(g,y)
\end{eulerprompt}
\begin{eulerformula}
\[
\left[ y=\frac{-\left(2\,{\it b_1}-2\,{\it a_1}\right)\,x+{\it b_2}  ^2+{\it b_1}^2-{\it a_2}^2-{\it a_1}^2}{2\,{\it b_2}-2\,{\it a_2}}   \right] 
\]
\end{eulerformula}
\begin{eulercomment}
Ini memang sama dengan tengah tegak lurus, yang dihitung dengan cara
yang sama sekali berbeda.
\end{eulercomment}
\begin{eulerprompt}
>$solve(getLineEquation(middlePerpendicular(A,B),x,y),y)
\end{eulerprompt}
\begin{eulerformula}
\[
\left[ y=\frac{-\left(2\,{\it b_1}-2\,{\it a_1}\right)\,x+{\it b_2}  ^2+{\it b_1}^2-{\it a_2}^2-{\it a_1}^2}{2\,{\it b_2}-2\,{\it a_2}}   \right] 
\]
\end{eulerformula}
\begin{eulerprompt}
>h &=getLineEquation(lineThrough(A,B),x,y);
>$solve(h,y)
\end{eulerprompt}
\begin{eulerformula}
\[
\left[ y=\frac{\left({\it b_2}-{\it a_2}\right)\,x-{\it a_1}\,  {\it b_2}+{\it a_2}\,{\it b_1}}{{\it b_1}-{\it a_1}} \right] 
\]
\end{eulerformula}
\begin{eulercomment}
Perhatikan hasil kali gradien garis g dan h adalah:

\end{eulercomment}
\begin{eulerformula}
\[
\frac{-(b_1-a_1)}{(b_2-a_2)}\times \frac{(b_2-a_2)}{(b_1-a_1)} = -1.
\]
\end{eulerformula}
\begin{eulercomment}
Artinya kedua garis tegak lurus.
\end{eulercomment}
\eulerheading{Contoh 3: Rumus Heron}
\begin{eulercomment}
Rumus Heron menyatakan bahwa luas segitiga dengan panjang sisi-sisi a,
b dan c adalah:

\end{eulercomment}
\begin{eulerformula}
\[
L = \sqrt{s(s-a)(s-b)(s-c)}\quad \text{ dengan } s=(a+b+c)/2.
\]
\end{eulerformula}
\begin{eulercomment}
Untuk membuktikan hal ini kita misalkan C(0,0), B(a,0) dan A(x,y),
b=AC, c=AB. Luas segitiga ABC adalah

\end{eulercomment}
\begin{eulerformula}
\[
L_{\triangle ABC}=\frac{1}{2}a\times y.
\]
\end{eulerformula}
\begin{eulercomment}
Nilai y didapat dengan menyelesaikan sistem persamaan:

\end{eulercomment}
\begin{eulerformula}
\[
x^2+y^2=b^2, \quad (x-a)^2+y^2=c^2.
\]
\end{eulerformula}
\begin{eulerprompt}
>setPlotRange(-1,10,-1,8); plotPoint([0,0], "C(0,0)"); plotPoint([5.5,0], "B(a,0)");  ...
> plotPoint([7.5,6], "A(x,y)");
>plotSegment([0,0],[5.5,0], "a",25); plotSegment([5.5,0],[7.5,6],"c",15);  ...
>plotSegment([0,0],[7.5,6],"b",25); 
>plotSegment([7.5,6],[7.5,0],"t=y",25):
\end{eulerprompt}
\eulerimg{27}{images/22305141017_Ardi Budi S_Geometri-032.png}
\begin{eulerprompt}
>sol &= solve([x^2+y^2=b^2,(x-a)^2+y^2=c^2],[x,y])
\end{eulerprompt}
\begin{euleroutput}
  
                   2    2    2
                - c  + b  + a
          [[x = --------------, y = 
                     2 a
            4      2  2      2  2    4      2  2    4
    sqrt(- c  + 2 b  c  + 2 a  c  - b  + 2 a  b  - a )
  - --------------------------------------------------], 
                           2 a
          2    2    2
       - c  + b  + a
  [x = --------------, y = 
            2 a
          4      2  2      2  2    4      2  2    4
  sqrt(- c  + 2 b  c  + 2 a  c  - b  + 2 a  b  - a )
  --------------------------------------------------]]
                         2 a
  
\end{euleroutput}
\begin{eulercomment}
Ekstrak solusi y
\end{eulercomment}
\begin{eulerprompt}
>ysol &= y with sol[2][2]; $'y=sqrt(factor(ysol^2))
\end{eulerprompt}
\begin{eulerformula}
\[
y=\frac{\sqrt{\left(-c+b+a\right)\,\left(c-b+a\right)\,\left(c+b-a  \right)}\,\sqrt{c+b+a}}{2\,a}
\]
\end{eulerformula}
\begin{eulercomment}
Kita mendapatkan formula Heron.
\end{eulercomment}
\begin{eulerprompt}
>function H(a,b,c) &= sqrt(factor((ysol*a/2)^2)); $'H(a,b,c)=H(a,b,c)
\end{eulerprompt}
\begin{eulerformula}
\[
H\left(a , b , c\right)=\frac{\sqrt{\left(-c+b+a\right)\,\left(c-b+  a\right)\,\left(c+b-a\right)}\,\sqrt{c+b+a}}{4}
\]
\end{eulerformula}
\begin{eulercomment}
Tentu saja, setiap segitiga persegi panjang adalah kasus yang
terkenal.
\end{eulercomment}
\begin{eulerprompt}
>$'Luas=H(2,5,6) // luas segitiga dengan panjang sisi-sisi 2, 5, 6
\end{eulerprompt}
\begin{eulerformula}
\[
{\it Luas}=\frac{3^{\frac{3}{2}}\,\sqrt{13}}{4}
\]
\end{eulerformula}
\begin{eulerprompt}
>H(3,4,5) //luas segitiga siku-siku dengan panjang sisi 3, 4, 5
\end{eulerprompt}
\begin{euleroutput}
  6
\end{euleroutput}
\begin{eulercomment}
Dan jelas juga, bahwa ini adalah segitiga dengan luas maksimal dan
kedua sisinya 3 dan 4.
\end{eulercomment}
\begin{eulerprompt}
>aspect (1.5); plot2d(&H(3,4,x),1,7): // Kurva luas segitiga sengan panjang sisi 3, 4, x (1<= x <=7)
\end{eulerprompt}
\eulerimg{17}{images/22305141017_Ardi Budi S_Geometri-036.png}
\begin{eulercomment}
Kasus umum juga berfungsi.
\end{eulercomment}
\begin{eulerprompt}
>$solve(diff(H(a,b,c)^2,c)=0,c)
\end{eulerprompt}
\begin{eulerformula}
\[
\left[ c=-\sqrt{b^2+a^2} , c=\sqrt{b^2+a^2} , c=0 \right] 
\]
\end{eulerformula}
\begin{eulercomment}
Sekarang mari kita cari himpunan semua titik di mana b + c = d untuk
beberapa konstanta d. Diketahui bahwa ini adalah elips.
\end{eulercomment}
\begin{eulerprompt}
>s1 &= subst(d-c,b,sol[2]); $s1
\end{eulerprompt}
\begin{eulerformula}
\[
\left[ x=\frac{\left(d-c\right)^2-c^2+a^2}{2\,a} , y=\frac{\sqrt{-  \left(d-c\right)^4+2\,c^2\,\left(d-c\right)^2+2\,a^2\,\left(d-c  \right)^2-c^4+2\,a^2\,c^2-a^4}}{2\,a} \right] 
\]
\end{eulerformula}
\begin{eulercomment}
Dan membuat persamaan seperti ini
\end{eulercomment}
\begin{eulerprompt}
>function fx(a,c,d) &= rhs(s1[1]); $fx(a,c,d), function fy(a,c,d) &= rhs(s1[2]); $fy(a,c,d)
\end{eulerprompt}
\begin{eulerformula}
\[
\frac{\sqrt{-\left(d-c\right)^4+2\,c^2\,\left(d-c\right)^2+2\,a^2\,  \left(d-c\right)^2-c^4+2\,a^2\,c^2-a^4}}{2\,a}
\]
\end{eulerformula}
\eulerimg{2}{images/22305141017_Ardi Budi S_Geometri-040-large.png}
\begin{eulercomment}
Sekarang kita bisa menggambar setnya. Sisi b bervariasi dari 1 hingga
4. Diketahui bahwa kita mendapatkan elips.
\end{eulercomment}
\begin{eulerprompt}
>aspect(1); plot2d(&fx(3,x,5),&fy(3,x,5),xmin=1,xmax=4,square=1):
\end{eulerprompt}
\eulerimg{27}{images/22305141017_Ardi Budi S_Geometri-041.png}
\begin{eulercomment}
Kita dapat memeriksa persamaan umum elips ini, yaitu.

\end{eulercomment}
\begin{eulerformula}
\[
\frac{(x-x_m)^2}{u^2}+\frac{(y-y_m)}{v^2}=1,
\]
\end{eulerformula}
\begin{eulercomment}
di mana (xm,ym) adalah pusat, dan u dan v adalah setengah sumbu.
\end{eulercomment}
\begin{eulerprompt}
>$ratsimp((fx(a,c,d)-a/2)^2/u^2+fy(a,c,d)^2/v^2 with [u=d/2,v=sqrt(d^2-a^2)/2])
\end{eulerprompt}
\begin{eulerformula}
\[
1
\]
\end{eulerformula}
\begin{eulercomment}
Kita melihat bahwa tinggi dan luas segitiga adalah maksimal untuk x =
0. Jadi luas segitiga dengan a + b + c = d adalah maksimal, jika sama
sisi. Kami ingin mendapatkan ini secara analitis.
\end{eulercomment}
\begin{eulerprompt}
>eqns &= [diff(H(a,b,d-(a+b))^2,a)=0,diff(H(a,b,d-(a+b))^2,b)=0]; $eqns
\end{eulerprompt}
\begin{eulerformula}
\[
\left[ \frac{d\,\left(d-2\,a\right)\,\left(d-2\,b\right)}{8}-\frac{  \left(-d+2\,b+2\,a\right)\,d\,\left(d-2\,b\right)}{8}=0 , \frac{d\,  \left(d-2\,a\right)\,\left(d-2\,b\right)}{8}-\frac{\left(-d+2\,b+2\,  a\right)\,d\,\left(d-2\,a\right)}{8}=0 \right] 
\]
\end{eulerformula}
\begin{eulercomment}
Kami mendapatkan beberapa minima, yang termasuk dalam segitiga dengan
satu sisi 0, dan solusi a=b=c=d/3.
\end{eulercomment}
\begin{eulerprompt}
>$solve(eqns,[a,b])
\end{eulerprompt}
\begin{eulerformula}
\[
\left[ \left[ a=\frac{d}{3} , b=\frac{d}{3} \right]  , \left[ a=0   , b=\frac{d}{2} \right]  , \left[ a=\frac{d}{2} , b=0 \right]  ,   \left[ a=\frac{d}{2} , b=\frac{d}{2} \right]  \right] 
\]
\end{eulerformula}
\begin{eulercomment}
Ada juga metode Lagrange, memaksimalkan H(a,b,c)\textasciicircum{}2 terhadap a+b+d=d.
\end{eulercomment}
\begin{eulerprompt}
>&solve([diff(H(a,b,c)^2,a)=la,diff(H(a,b,c)^2,b)=la, ...
>   diff(H(a,b,c)^2,c)=la,a+b+c=d],[a,b,c,la])
\end{eulerprompt}
\begin{euleroutput}
  
                       d      d
          [[a = 0, b = -, c = -, la = 0], 
                       2      2
       d             d                d      d
  [a = -, b = 0, c = -, la = 0], [a = -, b = -, c = 0, la = 0], 
       2             2                2      2
                              3
       d      d      d       d
  [a = -, b = -, c = -, la = ---]]
       3      3      3       108
  
\end{euleroutput}
\begin{eulercomment}
Kita bisa membuat plot situasinya.
\end{eulercomment}
\begin{eulercomment}
Pertama, atur poin di Maxima
\end{eulercomment}
\begin{eulerprompt}
>A &= at([x,y],sol[2]); $A
\end{eulerprompt}
\begin{eulerformula}
\[
\left[ \frac{-c^2+b^2+a^2}{2\,a} , \frac{\sqrt{-c^4+2\,b^2\,c^2+2\,  a^2\,c^2-b^4+2\,a^2\,b^2-a^4}}{2\,a} \right] 
\]
\end{eulerformula}
\begin{eulerprompt}
>B &= [0,0]; $B, C &= [a,0]; $C
\end{eulerprompt}
\begin{eulerformula}
\[
\left[ a , 0 \right] 
\]
\end{eulerformula}
\eulerimg{0}{images/22305141017_Ardi Budi S_Geometri-048-large.png}
\begin{eulercomment}
Kemudian atur rentang plot, dan plot poinnya.
\end{eulercomment}
\begin{eulerprompt}
>setPlotRange(0,5,-2,3); ...
>a=4; b=3; c=2; ...
>plotPoint(mxmeval("B"),"B"); plotPoint(mxmeval("C"),"C"); ...
>plotPoint(mxmeval("A"),"A"):
\end{eulerprompt}
\eulerimg{27}{images/22305141017_Ardi Budi S_Geometri-049.png}
\begin{eulercomment}
Plot segmennya.
\end{eulercomment}
\begin{eulerprompt}
>plotSegment(mxmeval("A"),mxmeval("C")); ...
>plotSegment(mxmeval("B"),mxmeval("C")); ...
>plotSegment(mxmeval("B"),mxmeval("A")):
\end{eulerprompt}
\eulerimg{27}{images/22305141017_Ardi Budi S_Geometri-050.png}
\begin{eulercomment}
Hitung tengah tegak lurus di Maxima.
\end{eulercomment}
\begin{eulerprompt}
>h &= middlePerpendicular(A,B); g &= middlePerpendicular(B,C);
\end{eulerprompt}
\begin{eulercomment}
Dan bagian tengah dari keliling.
\end{eulercomment}
\begin{eulerprompt}
>U &= lineIntersection(h,g);
\end{eulerprompt}
\begin{eulercomment}
Kita mendapatkan rumus untuk jari-jari lingkaran.
\end{eulercomment}
\begin{eulerprompt}
>&assume(a>0,b>0,c>0); $distance(U,B) | radcan
\end{eulerprompt}
\begin{eulerformula}
\[
\frac{i\,a\,b\,c}{\sqrt{c-b-a}\,\sqrt{c-b+a}\,\sqrt{c+b-a}\,\sqrt{c  +b+a}}
\]
\end{eulerformula}
\begin{eulercomment}
Mari kita tambahkan ini ke plot.
\end{eulercomment}
\begin{eulerprompt}
>plotPoint(U()); ...
>plotCircle(circleWithCenter(mxmeval("U"),mxmeval("distance(U,C)"))):
\end{eulerprompt}
\eulerimg{27}{images/22305141017_Ardi Budi S_Geometri-052.png}
\begin{eulercomment}
Menggunakan geometri, kita mendapatkan rumus sederhana

\end{eulercomment}
\begin{eulerformula}
\[
\frac{a}{\sin(\alpha)}=2r
\]
\end{eulerformula}
\begin{eulercomment}
untuk radius. Kita dapat memeriksa, apakah ini benar dengan Maxima.
Maxima akan menfaktorkannya hanya jika kita mengkuadratkannya.
\end{eulercomment}
\begin{eulerprompt}
>$c^2/sin(computeAngle(A,B,C))^2  | factor
\end{eulerprompt}
\begin{eulerformula}
\[
-\frac{4\,a^2\,b^2\,c^2}{\left(c-b-a\right)\,\left(c-b+a\right)\,  \left(c+b-a\right)\,\left(c+b+a\right)}
\]
\end{eulerformula}
\eulerheading{Contoh 4: Garis Euler dan Parabola}
\begin{eulercomment}
Garis euler adalah garis yang ditentukan dari segitiga yang tidak sama
sisi. Ini adalah garis tengah segitiga, dan melewati beberapa titik
penting yang ditentukan dari segitiga, termasuk pusat ortosentrum,
sirkumenter, pusat massa, titik Exeter, dan pusat lingkaran sembilan
titik segitiga.

Untuk demonstrasi, kami menghitung dan memplot garis Euler dalam
segitiga.

Pertama, kami menentukan sudut segitiga di Euler. Kami menggunakan
definisi, yang terlihat dalam ekspresi simbolik.
\end{eulercomment}
\begin{eulerprompt}
>A::=[-1,-1]; B::=[2,0]; C::=[1,2];
\end{eulerprompt}
\begin{eulercomment}
Untuk memplot objek geometris, kami menyiapkan area plot, dan
menambahkan poin ke dalamnya. Semua plot objek geometris ditambahkan
ke plot saat ini.
\end{eulercomment}
\begin{eulerprompt}
>setPlotRange(3); plotPoint(A,"A"); plotPoint(B,"B"); plotPoint(C,"C");
\end{eulerprompt}
\begin{eulercomment}
Kita juga bisa menambahkan sisi segitiga.
\end{eulercomment}
\begin{eulerprompt}
>plotSegment(A,B,""); plotSegment(B,C,""); plotSegment(C,A,""):
\end{eulerprompt}
\eulerimg{27}{images/22305141017_Ardi Budi S_Geometri-055.png}
\begin{eulercomment}
Berikut adalah luas segitiga menggunakan rumus determinan. Tentu saja
kita harus mengambil nilai absolut dari hasil ini.
\end{eulercomment}
\begin{eulerprompt}
>$areaTriangle(A,B,C)
\end{eulerprompt}
\begin{eulerformula}
\[
-\frac{7}{2}
\]
\end{eulerformula}
\begin{eulercomment}
Kita dapat menghitung koefisien dari sisi c.
\end{eulercomment}
\begin{eulerprompt}
>c &= lineThrough(A,B)
\end{eulerprompt}
\begin{euleroutput}
  
                              [- 1, 3, - 2]
  
\end{euleroutput}
\begin{eulercomment}
Dan juga dapatkan rumus untuk baris ini.
\end{eulercomment}
\begin{eulerprompt}
>$getLineEquation(c,x,y)
\end{eulerprompt}
\begin{eulerformula}
\[
3\,y-x=-2
\]
\end{eulerformula}
\begin{eulercomment}
Untuk bentuk Hesse, kita perlu menentukan titik, sehingga titik
tersebut berada di sisi positif dari bentuk Hesse. Memasukkan titik
menghasilkan jarak positif ke garis.
\end{eulercomment}
\begin{eulerprompt}
>$getHesseForm(c,x,y,C), $at(%,[x=C[1],y=C[2]])
\end{eulerprompt}
\begin{eulerformula}
\[
\frac{7}{\sqrt{10}}
\]
\end{eulerformula}
\eulerimg{1}{images/22305141017_Ardi Budi S_Geometri-059-large.png}
\begin{eulercomment}
Sekarang kita menghitung sirkit ABC.
\end{eulercomment}
\begin{eulerprompt}
>LL &= circleThrough(A,B,C); $getCircleEquation(LL,x,y)
\end{eulerprompt}
\begin{eulerformula}
\[
\left(y-\frac{5}{14}\right)^2+\left(x-\frac{3}{14}\right)^2=\frac{  325}{98}
\]
\end{eulerformula}
\begin{eulerprompt}
>O &= getCircleCenter(LL); $O
\end{eulerprompt}
\begin{eulerformula}
\[
\left[ \frac{3}{14} , \frac{5}{14} \right] 
\]
\end{eulerformula}
\begin{eulercomment}
Plot lingkaran dan pusatnya. Cu dan U adalah simbolik. Kami
mengevaluasi ekspresi ini untuk Euler.
\end{eulercomment}
\begin{eulerprompt}
>plotCircle(LL()); plotPoint(O(),"O"):
\end{eulerprompt}
\eulerimg{27}{images/22305141017_Ardi Budi S_Geometri-062.png}
\begin{eulercomment}
Kita dapat menghitung perpotongan ketinggian di ABC (orthocenter)
secara numerik dengan perintah berikut.
\end{eulercomment}
\begin{eulerprompt}
>H &= lineIntersection(perpendicular(A,lineThrough(C,B)),...
>  perpendicular(B,lineThrough(A,C))); $H
\end{eulerprompt}
\begin{eulerformula}
\[
\left[ \frac{11}{7} , \frac{2}{7} \right] 
\]
\end{eulerformula}
\begin{eulercomment}
Sekarang kita dapat menghitung garis Euler dari segitiga tersebut.
\end{eulercomment}
\begin{eulerprompt}
>el &= lineThrough(H,O); $getLineEquation(el,x,y)
\end{eulerprompt}
\begin{eulerformula}
\[
-\frac{19\,y}{14}-\frac{x}{14}=-\frac{1}{2}
\]
\end{eulerformula}
\begin{eulercomment}
Tambahkan ke plot kita.
\end{eulercomment}
\begin{eulerprompt}
>plotPoint(H(),"H"); plotLine(el(),"Garis Euler"):
\end{eulerprompt}
\eulerimg{27}{images/22305141017_Ardi Budi S_Geometri-065.png}
\begin{eulercomment}
Pusat gravitasi harus berada di garis ini.
\end{eulercomment}
\begin{eulerprompt}
>M &= (A+B+C)/3; $getLineEquation(el,x,y) with [x=M[1],y=M[2]]
\end{eulerprompt}
\begin{eulerformula}
\[
-\frac{1}{2}=-\frac{1}{2}
\]
\end{eulerformula}
\begin{eulerprompt}
>plotPoint(M(),"M"): // titik berat
\end{eulerprompt}
\eulerimg{27}{images/22305141017_Ardi Budi S_Geometri-067.png}
\begin{eulercomment}
Teorinya mengatakan bahwa MH=2*MO. Kita perlu menyederhanakan dengan
radcan untuk mencapai ini.
\end{eulercomment}
\begin{eulerprompt}
>$distance(M,H)/distance(M,O)|radcan
\end{eulerprompt}
\begin{eulerformula}
\[
2
\]
\end{eulerformula}
\begin{eulercomment}
Fungsinya termasuk fungsi untuk sudut juga.
\end{eulercomment}
\begin{eulerprompt}
>$computeAngle(A,C,B), degprint(%())
\end{eulerprompt}
\begin{eulerformula}
\[
\arccos \left(\frac{4}{\sqrt{5}\,\sqrt{13}}\right)
\]
\end{eulerformula}
\begin{euleroutput}
  60°15'18.43''
\end{euleroutput}
\begin{eulercomment}
Persamaan untuk pusat lingkaran tidak terlalu bagus.
\end{eulercomment}
\begin{eulerprompt}
>Q &= lineIntersection(angleBisector(A,C,B),angleBisector(C,B,A))|radcan; $Q
\end{eulerprompt}
\begin{eulerformula}
\[
\left[ \frac{\left(2^{\frac{3}{2}}+1\right)\,\sqrt{5}\,\sqrt{13}-15  \,\sqrt{2}+3}{14} , \frac{\left(\sqrt{2}-3\right)\,\sqrt{5}\,\sqrt{  13}+5\,2^{\frac{3}{2}}+5}{14} \right] 
\]
\end{eulerformula}
\begin{eulercomment}
Mari kita hitung juga ekspresi jari-jari lingkaran yang tertulis.
\end{eulercomment}
\begin{eulerprompt}
>r &= distance(Q,projectToLine(Q,lineThrough(A,B)))|ratsimp; $r
\end{eulerprompt}
\begin{eulerformula}
\[
\frac{\sqrt{\left(-41\,\sqrt{2}-31\right)\,\sqrt{5}\,\sqrt{13}+115  \,\sqrt{2}+614}}{7\,\sqrt{2}}
\]
\end{eulerformula}
\begin{eulerprompt}
>LD &=  circleWithCenter(Q,r); // Lingkaran dalam
\end{eulerprompt}
\begin{eulercomment}
Mari kita tambahkan ini ke plot.
\end{eulercomment}
\begin{eulerprompt}
>color(5); plotCircle(LD()):
\end{eulerprompt}
\eulerimg{27}{images/22305141017_Ardi Budi S_Geometri-072.png}
\eulersubheading{Parabola}
\begin{eulercomment}
Selanjutnya akan dicari persamaan tempat kedudukan titik-titik yang berjarak sama ke titik C
dan ke garis AB.
\end{eulercomment}
\begin{eulerprompt}
>p &= getHesseForm(lineThrough(A,B),x,y,C)-distance([x,y],C); $p='0
\end{eulerprompt}
\begin{eulerformula}
\[
\frac{3\,y-x+2}{\sqrt{10}}-\sqrt{\left(2-y\right)^2+\left(1-x  \right)^2}=0
\]
\end{eulerformula}
\begin{eulercomment}
Persamaan tersebut dapat digambar menjadi satu dengan gambar sebelumnya.
\end{eulercomment}
\begin{eulerprompt}
>plot2d(p,level=0,add=1,contourcolor=6):
\end{eulerprompt}
\eulerimg{27}{images/22305141017_Ardi Budi S_Geometri-074.png}
\begin{eulercomment}
Ini seharusnya menjadi beberapa fungsi, tetapi pemecah default Maxima
dapat menemukan solusi hanya, jika persamaan kita kuadratkan.
Akibatnya, kami mendapatkan solusi palsu.
\end{eulercomment}
\begin{eulerprompt}
>akar &= solve(getHesseForm(lineThrough(A,B),x,y,C)^2-distance([x,y],C)^2,y)
\end{eulerprompt}
\begin{euleroutput}
  
          [y = - 3 x - sqrt(70) sqrt(9 - 2 x) + 26, 
                                y = - 3 x + sqrt(70) sqrt(9 - 2 x) + 26]
  
\end{euleroutput}
\begin{eulercomment}
Solusi pertama adalah

\end{eulercomment}
\begin{eulerformula}
\[
y=-3\,x-\sqrt{70}\,\sqrt{9-2\,x}+26
\]
\end{eulerformula}
\begin{eulercomment}
Menambahkan solusi pertama ke pertunjukkan plot, bahwa itu memang
jalan yang kita cari. Teori mengatakan kepada kita bajwa itu adalah
parabola yang diputar.
\end{eulercomment}
\begin{eulerprompt}
>plot2d(&rhs(akar[1]),add=1):
\end{eulerprompt}
\eulerimg{27}{images/22305141017_Ardi Budi S_Geometri-076.png}
\begin{eulerprompt}
>function g(x) &= rhs(akar[1]); $'g(x)= g(x)// fungsi yang mendefinisikan kurva di atas
\end{eulerprompt}
\begin{eulerformula}
\[
g\left(x\right)=-3\,x-\sqrt{70}\,\sqrt{9-2\,x}+26
\]
\end{eulerformula}
\begin{eulerprompt}
>T &=[-1, g(-1)]; // ambil sebarang titik pada kurva tersebut
>dTC &= distance(T,C); $fullratsimp(dTC), $float(%) // jarak T ke C
\end{eulerprompt}
\begin{eulerformula}
\[
2.135605779339061
\]
\end{eulerformula}
\eulerimg{0}{images/22305141017_Ardi Budi S_Geometri-079-large.png}
\begin{eulerprompt}
>U &= projectToLine(T,lineThrough(A,B)); $U // proyeksi T pada garis AB 
\end{eulerprompt}
\begin{eulerformula}
\[
\left[ \frac{80-3\,\sqrt{11}\,\sqrt{70}}{10} , \frac{20-\sqrt{11}\,  \sqrt{70}}{10} \right] 
\]
\end{eulerformula}
\begin{eulerprompt}
>dU2AB &= distance(T,U); $fullratsimp(dU2AB), $float(%) // jatak T ke AB
\end{eulerprompt}
\begin{eulerformula}
\[
2.135605779339061
\]
\end{eulerformula}
\eulerimg{0}{images/22305141017_Ardi Budi S_Geometri-082-large.png}
\begin{eulercomment}
Ternyata jarak T ke C sama dengan jarak T ke AB. Coba Anda pilih titik T yang lain dan
ulangi perhitungan-perhitungan di atas untuk menunjukkan bahwa hasilnya juga sama.
\end{eulercomment}
\begin{eulercomment}

\begin{eulercomment}
\eulerheading{Contoh 5: Trigonometri Rasional}
\begin{eulercomment}
Ini terinspirasi oleh ceramah N.J.Wildberger. Dalam bukunya "Proporsi
Agung", Wildberger mengusulkan untuk menggantikan pengertian klasik
tentang jarak dan sudut dengan kuadransi dan penyebaran. Dengan
menggunakan ini, memang mungkin untuk menghindari fungsi trigonometri
dalam banyak contoh, dan tetap "rasional".

Berikut ini, saya memperkenalkan konsep, dan memecahkan beberapa
masalah. Saya menggunakan perhitungan simbolik Maxima di sini, yang
menyembunyikan keuntungan utama dari trigonometri rasional bahwa
perhitungan dapat dilakukan dengan kertas dan pensil saja. Anda
diundang untuk memeriksa hasil tanpa komputer.

Intinya adalah bahwa perhitungan rasional simbolis sering kali
menghasilkan hasil yang sederhana. Sebaliknya, trigonometri klasik
menghasilkan hasil trigonometri yang rumit, yang mengevaluasi ke
pendekatan numerik saja.
\end{eulercomment}
\begin{eulerprompt}
>load geometry;
\end{eulerprompt}
\begin{eulercomment}
Untuk pendahuluan pertama, kami menggunakan segitiga persegi panjang
dengan proporsi Mesir terkenal 3, 4 dan 5. Perintah berikut adalah
perintah Euler untuk memplot geometri bidang yang terdapat dalam file
Euler "geometry.e".
\end{eulercomment}
\begin{eulerprompt}
>C&:=[0,0]; A&:=[4,0]; B&:=[0,3]; ...
>setPlotRange(-1,5,-1,5); ...
>plotPoint(A,"A"); plotPoint(B,"B"); plotPoint(C,"C"); ...
>plotSegment(B,A,"c"); plotSegment(A,C,"b"); plotSegment(C,B,"a"); ...
>insimg(30);
\end{eulerprompt}
\eulerimg{27}{images/22305141017_Ardi Budi S_Geometri-083.png}
\begin{eulercomment}
Tentu saja,

\end{eulercomment}
\begin{eulerformula}
\[
\sin(w_a)=\frac{a}{c},
\]
\end{eulerformula}
\begin{eulercomment}
di mana wa adalah sudut di A. Cara biasa untuk menghitung sudut ini,
adalah dengan melakukan invers dari fungsi sinus. Hasilnya adalah
sudut yang tidak dapat dicerna, yang hanya dapat dicetak secara
perkiraan.
\end{eulercomment}
\begin{eulerprompt}
>wa := arcsin(3/5); degprint(wa)
\end{eulerprompt}
\begin{euleroutput}
  36°52'11.63''
\end{euleroutput}
\begin{eulercomment}
Trigonometri rasional mencoba menghindari hal ini.

Pengertian pertama dari trigonometri rasional adalah kuadran, yang
menggantikan jarak. Faktanya, itu hanyalah kuadrat jarak. Berikut ini,
a, b, dan c menunjukkan kuadran sisi-sisinya.

Teorema Pythogoras menjadi a+b=c lalu.
\end{eulercomment}
\begin{eulerprompt}
>a &= 3^2; b &= 4^2; c &= 5^2; &a+b=c
\end{eulerprompt}
\begin{euleroutput}
  
                                 25 = 25
  
\end{euleroutput}
\begin{eulercomment}
Gagasan kedua dari trigonometri rasional adalah penyebarannya. Spread
mengukur bukaan antar baris. Ini adalah 0, jika garis sejajar, dan 1,
jika garis persegi panjang. Ini adalah kuadrat dari sinus sudut antara
dua garis.

Penyebaran garis AB dan AC pada gambar di atas didefinisikan sebagai

\end{eulercomment}
\begin{eulerformula}
\[
s_a = \sin(\alpha)^2 = \frac{a}{c},
\]
\end{eulerformula}
\begin{eulercomment}
di mana a dan c adalah kuadrat dari segitiga persegi panjang mana pun
dengan satu sudut di A.
\end{eulercomment}
\begin{eulerprompt}
>sa &= a/c; $sa
\end{eulerprompt}
\begin{eulerformula}
\[
\frac{9}{25}
\]
\end{eulerformula}
\begin{eulercomment}
Ini lebih mudah dihitung daripada sudut, tentu saja. Tetapi Anda
kehilangan properti yang sudut dapat ditambahkan dengan mudah.

Tentu saja, kita dapat mengubah nilai perkiraan sudut wa menjadi
sprad, dan mencetaknya sebagai pecahan.
\end{eulercomment}
\begin{eulerprompt}
>fracprint(sin(wa)^2)
\end{eulerprompt}
\begin{euleroutput}
  9/25
\end{euleroutput}
\begin{eulercomment}
Hukum cosinus dari trgonometri klasik diterjemahkan menjadi "hukum
silang" berikut.

\end{eulercomment}
\begin{eulerformula}
\[
(c+b-a)^2 = 4 b c \, (1-s_a)
\]
\end{eulerformula}
\begin{eulercomment}
Di sini a, b, dan c adalah kuadran dari sisi-sisi segitiga, dan sa
adalah sebaran di sudut A. Sisi a, seperti biasa, berlawanan dengan
sudut A.

Hukum ini diimplementasikan dalam file geometry.e yang kami muat ke
Euler.
\end{eulercomment}
\begin{eulerprompt}
>$crosslaw(aa,bb,cc,saa)
\end{eulerprompt}
\begin{eulerformula}
\[
\left({\it cc}+{\it bb}-{\it aa}\right)^2=4\,{\it bb}\,{\it cc}\,  \left(1-{\it saa}\right)
\]
\end{eulerformula}
\begin{eulercomment}
Dalam kasus kami, kita mendapatkan
\end{eulercomment}
\begin{eulerprompt}
>$crosslaw(a,b,c,sa)
\end{eulerprompt}
\begin{eulerformula}
\[
1024=1024
\]
\end{eulerformula}
\begin{eulercomment}
Mari kita gunakan crosslaw ini untuk mencari sebaran di A. Untuk
melakukan ini, kita menghasilkan crosslaw untuk kuadran a, b, dan c,
dan menyelesaikannya untuk sebaran yang tidak diketahui sa.

Anda dapat melakukan ini dengan tangan dengan mudah, tetapi saya
menggunakan Maxima. Tentu saja, kami mendapatkan hasilnya, kami sudah
mendapatkannya.
\end{eulercomment}
\begin{eulerprompt}
>$crosslaw(a,b,c,x), $solve(%,x)
\end{eulerprompt}
\begin{eulerformula}
\[
\left[ x=\frac{9}{25} \right] 
\]
\end{eulerformula}
\eulerimg{1}{images/22305141017_Ardi Budi S_Geometri-091-large.png}
\begin{eulercomment}
Kami sudah tahu ini. Definisi penyebaran adalah kasus khusus dari
hukum lintas hukum.

Kita juga bisa menyelesaikan ini untuk umum a, b, c. Hasilnya adalah
rumus yang menghitung sebaran sudut segitiga berdasarkan kuadran
ketiga sisinya.
\end{eulercomment}
\begin{eulerprompt}
>$solve(crosslaw(aa,bb,cc,x),x)
\end{eulerprompt}
\begin{eulerformula}
\[
\left[ x=\frac{-{\it cc}^2-\left(-2\,{\it bb}-2\,{\it aa}\right)\,  {\it cc}-{\it bb}^2+2\,{\it aa}\,{\it bb}-{\it aa}^2}{4\,{\it bb}\,  {\it cc}} \right] 
\]
\end{eulerformula}
\begin{eulercomment}
Kita bisa membuat fungsi dari hasilnya. Fungsi seperti itu sudah
ditentukan dalam file geometry.e Euler.
\end{eulercomment}
\begin{eulerprompt}
>$spread(a,b,c)
\end{eulerprompt}
\begin{eulerformula}
\[
\frac{9}{25}
\]
\end{eulerformula}
\begin{eulercomment}
Sebagai contoh, kita bisa menggunakannya untuk menghitung sudut
segitiga bersisi

\end{eulercomment}
\begin{eulerformula}
\[
a, \quad a, \quad \frac{4a}{7}
\]
\end{eulerformula}
\begin{eulercomment}
Hasilnya rasional, yang tidak mudah didapat jika kita menggunakan
trigonometri klasik.
\end{eulercomment}
\begin{eulerprompt}
>$spread(a,a,4*a/7)
\end{eulerprompt}
\begin{eulerformula}
\[
\frac{6}{7}
\]
\end{eulerformula}
\begin{eulercomment}
Ini adalah sudut dalam derajat.
\end{eulercomment}
\begin{eulerprompt}
>degprint(arcsin(sqrt(6/7)))
\end{eulerprompt}
\begin{euleroutput}
  67°47'32.44''
\end{euleroutput}
\eulersubheading{Contoh lain}
\begin{eulercomment}
Sekarang, mari kita coba contoh yang lebih canggih.

Kami mengatur tiga sudut segitiga sebagai berikut.
\end{eulercomment}
\begin{eulerprompt}
>A&:=[1,2]; B&:=[4,3]; C&:=[0,4]; ...
>setPlotRange(-1,5,1,7); ...
>plotPoint(A,"A"); plotPoint(B,"B"); plotPoint(C,"C"); ...
>plotSegment(B,A,"c"); plotSegment(A,C,"b"); plotSegment(C,B,"a"); ...
>insimg;
\end{eulerprompt}
\eulerimg{27}{images/22305141017_Ardi Budi S_Geometri-096.png}
\begin{eulercomment}
Menggunakan Pythogoras, mudah untuk menghitung jarak antara dua titik.
Saya pertama kali menggunakan jarak fungsi file Euler untuk geometri.
Jarak fungsi menggunakan geometri klasik.
\end{eulercomment}
\begin{eulerprompt}
>$distance(A,B)
\end{eulerprompt}
\begin{eulerformula}
\[
\sqrt{10}
\]
\end{eulerformula}
\begin{eulercomment}
Euler juga memiliki fungsi kuadrans antara dua titik.

Dalam contoh berikut, karena c + b bukan a, segitiga tidak persegi
panjang.
\end{eulercomment}
\begin{eulerprompt}
>c &= quad(A,B); $c, b &= quad(A,C); $b, a &= quad(B,C); $a,
\end{eulerprompt}
\begin{eulerformula}
\[
17
\]
\end{eulerformula}
\eulerimg{0}{images/22305141017_Ardi Budi S_Geometri-099-large.png}
\eulerimg{0}{images/22305141017_Ardi Budi S_Geometri-100-large.png}
\begin{eulercomment}
Pertama, mari kita hitung sudut tradisional. Fungsi computeAngle
menggunakan metode biasa berdasarkan perkalian titik dari dua vektor.
Hasilnya adalah beberapa pendekatan floating point.

\end{eulercomment}
\begin{eulerformula}
\[
A=<1,2>\quad B=<4,3>,\quad C=<0,4>
\]
\end{eulerformula}
\begin{eulerformula}
\[
\mathbf{a}=C-B=<-4,1>,\quad \mathbf{c}=A-B=<-3,-1>,\quad \beta=\angle ABC
\]
\end{eulerformula}
\begin{eulerformula}
\[
\mathbf{a}.\mathbf{c}=|\mathbf{a}|.|\mathbf{c}|\cos \beta
\]
\end{eulerformula}
\begin{eulerformula}
\[
\cos \angle ABC =\cos\beta=\frac{\mathbf{a}.\mathbf{c}}{|\mathbf{a}|.|\mathbf{c}|}=\frac{12-1}{\sqrt{17}\sqrt{10}}=\frac{11}{\sqrt{17}\sqrt{10}}
\]
\end{eulerformula}
\begin{eulerprompt}
>wb &= computeAngle(A,B,C); $wb, $(wb/pi*180)()
\end{eulerprompt}
\begin{eulerformula}
\[
\arccos \left(\frac{11}{\sqrt{10}\,\sqrt{17}}\right)
\]
\end{eulerformula}
\begin{euleroutput}
  32.4711922908
\end{euleroutput}
\begin{eulercomment}
Menggunakan pensil dan kertas, kita bisa melakukan hal yang sama
dengan hukum silang. Kami memasukkan kuadran a, b, dan c ke dalam
hukum silang dan menyelesaikan untuk x.
\end{eulercomment}
\begin{eulerprompt}
>$crosslaw(a,b,c,x), $solve(%,x),
\end{eulerprompt}
\begin{eulerformula}
\[
\left[ x=\frac{49}{50} \right] 
\]
\end{eulerformula}
\eulerimg{1}{images/22305141017_Ardi Budi S_Geometri-107-large.png}
\begin{eulercomment}
Artinya, fungsi penyebaran yang didefinisikan dalam "geometry.e".
\end{eulercomment}
\begin{eulerprompt}
>sb &= spread(b,a,c); $sb
\end{eulerprompt}
\begin{eulerformula}
\[
\frac{49}{170}
\]
\end{eulerformula}
\begin{eulercomment}
Maxima mendapatkan hasil yang sama dengan menggunakan trigonometri
biasa, jika kita memaksakannya. Itu menyelesaikan istilah sin (arccos
(...)) menjadi hasil pecahan. Kebanyakan siswa tidak dapat melakukan
ini.
\end{eulercomment}
\begin{eulerprompt}
>$sin(computeAngle(A,B,C))^2
\end{eulerprompt}
\begin{eulerformula}
\[
\frac{49}{170}
\]
\end{eulerformula}
\begin{eulercomment}
Setelah kita mendapatkan sebaran di B, kita bisa menghitung tinggi ha
di sisi a. Ingat bahwa

\end{eulercomment}
\begin{eulerformula}
\[
s_b=\frac{h_a}{c}
\]
\end{eulerformula}
\begin{eulercomment}
Menurut definisi.
\end{eulercomment}
\begin{eulerprompt}
>ha &= c*sb; $ha
\end{eulerprompt}
\begin{eulerformula}
\[
\frac{49}{17}
\]
\end{eulerformula}
\begin{eulercomment}
Gambar berikut telah diproduksi dengan program geometri C.a.R., yang
dapat menggambar kuadran dan menyebar.

\end{eulercomment}
\eulerimg{15}{images/22305141017_Ardi Budi S_Geometri-112.png}
\begin{eulercomment}

Menurut definisi, panjang ha adalah akar kuadrat dari kuadrannya.
\end{eulercomment}
\begin{eulerprompt}
>$sqrt(ha)
\end{eulerprompt}
\begin{eulerformula}
\[
\frac{7}{\sqrt{17}}
\]
\end{eulerformula}
\begin{eulercomment}
Sekarang kita bisa menghitung luas segitiga. Jangan lupa, bahwa kita
berurusan dengan kuadran!
\end{eulercomment}
\begin{eulerprompt}
>$sqrt(ha)*sqrt(a)/2
\end{eulerprompt}
\begin{eulerformula}
\[
\frac{7}{2}
\]
\end{eulerformula}
\begin{eulercomment}
Rumus determinan yang biasa menghasilkan hasil yang sama.
\end{eulercomment}
\begin{eulerprompt}
>$areaTriangle(B,A,C)
\end{eulerprompt}
\begin{eulerformula}
\[
\frac{7}{2}
\]
\end{eulerformula}
\eulersubheading{Formula Heron}
\begin{eulercomment}
Sekarang, mari kita selesaikan masalah ini secara umum!
\end{eulercomment}
\begin{eulerprompt}
>&remvalue(a,b,c,sb,ha);
\end{eulerprompt}
\begin{eulercomment}
Pertama-tama kita menghitung spread di B untuk segitiga dengan sisi a,
b, dan c. Kemudian kami menghitung luas area yang dikuadratkan
("kuadrea"?), Memfaktorkannya dengan Maxima, dan kami mendapatkan
rumus Heron yang terkenal.

Memang, ini sulit dilakukan dengan pensil dan kertas.
\end{eulercomment}
\begin{eulerprompt}
>$spread(b^2,c^2,a^2), $factor(%*c^2*a^2/4)
\end{eulerprompt}
\begin{eulerformula}
\[
\frac{\left(-c+b+a\right)\,\left(c-b+a\right)\,\left(c+b-a\right)\,  \left(c+b+a\right)}{16}
\]
\end{eulerformula}
\eulerimg{1}{images/22305141017_Ardi Budi S_Geometri-117-large.png}
\eulersubheading{Aturan Triple Spread}
\begin{eulercomment}
Kerugian dari spread adalah bahwa mereka tidak lagi hanya menambahkan
sudut serupa.

Namun, tiga sebaran segitiga memenuhi aturan "penyebaran rangkap tiga"
berikut.
\end{eulercomment}
\begin{eulerprompt}
>&remvalue(sa,sb,sc); $triplespread(sa,sb,sc)
\end{eulerprompt}
\begin{eulerformula}
\[
\left({\it sc}+{\it sb}+{\it sa}\right)^2=2\,\left({\it sc}^2+  {\it sb}^2+{\it sa}^2\right)+4\,{\it sa}\,{\it sb}\,{\it sc}
\]
\end{eulerformula}
\begin{eulercomment}
Aturan ini berlaku untuk tiga sudut yang bertambah menjadi 180°.

\end{eulercomment}
\begin{eulerformula}
\[
\alpha+\beta+\gamma=\pi
\]
\end{eulerformula}
\begin{eulercomment}
Sejak penyebaran

\end{eulercomment}
\begin{eulerformula}
\[
\alpha, \pi-\alpha
\]
\end{eulerformula}
\begin{eulercomment}
sama, aturan penyebaran tiga kali lipat juga benar, jika

\end{eulercomment}
\begin{eulerformula}
\[
\alpha+\beta=\gamma
\]
\end{eulerformula}
\begin{eulercomment}
Karena penyebaran sudut negatif adalah sama, aturan penyebaran tiga
kali lipat juga berlaku, jika

\end{eulercomment}
\begin{eulerformula}
\[
\alpha+\beta+\gamma=0
\]
\end{eulerformula}
\begin{eulercomment}
Misalnya, kita dapat menghitung sebaran sudut 60°. Ini 3/4. Persamaan
memiliki solusi kedua, di mana semua spread adalah 0.
\end{eulercomment}
\begin{eulerprompt}
>$solve(triplespread(x,x,x),x)
\end{eulerprompt}
\begin{eulerformula}
\[
\left[ x=\frac{3}{4} , x=0 \right] 
\]
\end{eulerformula}
\begin{eulercomment}
Sebaran 90° jelaslah 1. Jika dua sudut dijumlahkan menjadi 90°,
penyebarannya menyelesaikan persamaan penyebaran rangkap tiga dengan
a, b, 1. Dengan perhitungan berikut kita mendapatkan a+b=1.
\end{eulercomment}
\begin{eulerprompt}
>$triplespread(x,y,1), $solve(%,x)
\end{eulerprompt}
\begin{eulerformula}
\[
\left[ x=1-y \right] 
\]
\end{eulerformula}
\eulerimg{0}{images/22305141017_Ardi Budi S_Geometri-125-large.png}
\begin{eulercomment}
Karena penyebaran 180°-t sama dengan penyebaran t, rumus penyebaran
rangkap tiga juga berlaku, jika satu sudut adalah jumlah atau
perbedaan dari dua sudut lainnya.

Jadi kita bisa menemukan sebaran sudut berlipat ganda. Perhatikan
bahwa ada dua solusi lagi. Kami menjadikan ini sebuah fungsi.
\end{eulercomment}
\begin{eulerprompt}
>$solve(triplespread(a,a,x),x), function doublespread(a) &= factor(rhs(%[1]))
\end{eulerprompt}
\begin{eulerformula}
\[
\left[ x=4\,a-4\,a^2 , x=0 \right] 
\]
\end{eulerformula}
\begin{euleroutput}
  
                              - 4 (a - 1) a
  
\end{euleroutput}
\eulersubheading{Pembagi Sudut}
\begin{eulercomment}
Ini situasinya, kita sudah tahu.
\end{eulercomment}
\begin{eulerprompt}
>C&:=[0,0]; A&:=[4,0]; B&:=[0,3]; ...
>setPlotRange(-1,5,-1,5); ...
>plotPoint(A,"A"); plotPoint(B,"B"); plotPoint(C,"C"); ...
>plotSegment(B,A,"c"); plotSegment(A,C,"b"); plotSegment(C,B,"a"); ...
>insimg;
\end{eulerprompt}
\eulerimg{27}{images/22305141017_Ardi Budi S_Geometri-127.png}
\begin{eulercomment}
Mari kita hitung panjang bisektor sudut pada A. Tapi kita ingin
menyelesaikannya untuk umum a, b, c.
\end{eulercomment}
\begin{eulerprompt}
>&remvalue(a,b,c);
\end{eulerprompt}
\begin{eulercomment}
Jadi pertama-tama kita menghitung sebaran sudut terbagi di A,
menggunakan rumus sebaran rangkap tiga.

Masalah dengan rumus ini muncul lagi. Ini memiliki dua solusi. Kami
harus memilih yang benar. Solusi lainnya mengacu pada sudut terbagi
180°-wa.
\end{eulercomment}
\begin{eulerprompt}
>$triplespread(x,x,a/(a+b)), $solve(%,x), sa2 &= rhs(%[1]); $sa2
\end{eulerprompt}
\begin{eulerformula}
\[
\frac{-\sqrt{b}\,\sqrt{b+a}+b+a}{2\,b+2\,a}
\]
\end{eulerformula}
\eulerimg{2}{images/22305141017_Ardi Budi S_Geometri-129-large.png}
\eulerimg{1}{images/22305141017_Ardi Budi S_Geometri-130-large.png}
\begin{eulercomment}
Mari kita periksa persegi panjang Mesir.
\end{eulercomment}
\begin{eulerprompt}
>$sa2 with [a=3^2,b=4^2]
\end{eulerprompt}
\begin{eulerformula}
\[
\frac{1}{10}
\]
\end{eulerformula}
\begin{eulercomment}
Kami dapat mencetak sudut di Euler, setelah mentransfer penyebaran ke
radian.
\end{eulercomment}
\begin{eulerprompt}
>wa2 := arcsin(sqrt(1/10)); degprint(wa2)
\end{eulerprompt}
\begin{euleroutput}
  18°26'5.82''
\end{euleroutput}
\begin{eulercomment}
Titik P adalah perpotongan dari garis bagi sudut dengan sumbu y.
\end{eulercomment}
\begin{eulerprompt}
>P := [0,tan(wa2)*4]
\end{eulerprompt}
\begin{euleroutput}
  [0,  1.33333]
\end{euleroutput}
\begin{eulerprompt}
>plotPoint(P,"P"); plotSegment(A,P):
\end{eulerprompt}
\eulerimg{27}{images/22305141017_Ardi Budi S_Geometri-132.png}
\begin{eulercomment}
Mari kita periksa sudut dalam contoh spesifik kita.
\end{eulercomment}
\begin{eulerprompt}
>computeAngle(C,A,P), computeAngle(P,A,B)
\end{eulerprompt}
\begin{euleroutput}
  0.321750554397
  0.321750554397
\end{euleroutput}
\begin{eulercomment}
Sekarang kita menghitung panjang bisektor AP.

Kita menggunakan teorema sinus di segitiga APC. Teorema ini menyatakan
bahwa

\end{eulercomment}
\begin{eulerformula}
\[
\frac{BC}{\sin(w_a)} = \frac{AC}{\sin(w_b)} = \frac{AB}{\sin(w_c)}
\]
\end{eulerformula}
\begin{eulercomment}
memegang di segitiga apa pun. Persegi itu, itu diterjemahkan ke dalam
apa yang disebut "hukum penyebaran"

\end{eulercomment}
\begin{eulerformula}
\[
\frac{a}{s_a} = \frac{b}{s_b} = \frac{c}{s_b}
\]
\end{eulerformula}
\begin{eulercomment}
dimana a, b, c menunjukkan qudrance.

Karena BPA sebaran adalah 1-sa2, kita dapatkan darinya bisa / 1 = b /
(1-sa2) dan dapat menghitung bisa (kuadran garis-garis).
\end{eulercomment}
\begin{eulerprompt}
>&factor(ratsimp(b/(1-sa2))); bisa &= %; $bisa
\end{eulerprompt}
\begin{eulerformula}
\[
\frac{2\,b\,\left(b+a\right)}{\sqrt{b}\,\sqrt{b+a}+b+a}
\]
\end{eulerformula}
\begin{eulercomment}
Mari kita periksa rumus ini untuk nilai Mesir kita.
\end{eulercomment}
\begin{eulerprompt}
>sqrt(mxmeval("at(bisa,[a=3^2,b=4^2])")), distance(A,P)
\end{eulerprompt}
\begin{euleroutput}
  4.21637021356
  4.21637021356
\end{euleroutput}
\begin{eulercomment}
Kami juga dapat menghitung P menggunakan rumus spread.
\end{eulercomment}
\begin{eulerprompt}
>py&=factor(ratsimp(sa2*bisa)); $py
\end{eulerprompt}
\begin{eulerformula}
\[
-\frac{b\,\left(\sqrt{b}\,\sqrt{b+a}-b-a\right)}{\sqrt{b}\,\sqrt{b+  a}+b+a}
\]
\end{eulerformula}
\begin{eulercomment}
Nilainya sama dengan yang kita dapatkan dengan rumus trigonometri.
\end{eulercomment}
\begin{eulerprompt}
>sqrt(mxmeval("at(py,[a=3^2,b=4^2])"))
\end{eulerprompt}
\begin{euleroutput}
  1.33333333333
\end{euleroutput}
\eulersubheading{Sudut Akord}
\begin{eulercomment}
Perhatikan situasi berikut.
\end{eulercomment}
\begin{eulerprompt}
>setPlotRange(1.2); ...
>color(1); plotCircle(circleWithCenter([0,0],1)); ...
>A:=[cos(1),sin(1)]; B:=[cos(2),sin(2)]; C:=[cos(6),sin(6)]; ...
>plotPoint(A,"A"); plotPoint(B,"B"); plotPoint(C,"C"); ...
>color(3); plotSegment(A,B,"c"); plotSegment(A,C,"b"); plotSegment(C,B,"a"); ...
>color(1); O:=[0,0];  plotPoint(O,"0"); ...
>plotSegment(A,O); plotSegment(B,O); plotSegment(C,O,"r"); ...
>insimg;
\end{eulerprompt}
\eulerimg{27}{images/22305141017_Ardi Budi S_Geometri-137.png}
\begin{eulercomment}
Kita bisa menggunakan Maxima untuk menyelesaikan rumus sebaran rangkap
tiga untuk sudut di pusat O untuk r. Jadi kita mendapatkan rumus untuk
jari-jari kuadrat dari keliling dalam hal kuadrat sisi.

Kali ini, Maxima menghasilkan beberapa angka nol yang kompleks, yang
kita abaikan.
\end{eulercomment}
\begin{eulerprompt}
>&remvalue(a,b,c,r); // hapus nilai-nilai sebelumnya untuk perhitungan baru
>rabc &= rhs(solve(triplespread(spread(b,r,r),spread(a,r,r),spread(c,r,r)),r)[4]); $rabc
\end{eulerprompt}
\begin{eulerformula}
\[
-\frac{a\,b\,c}{c^2-2\,b\,c+a\,\left(-2\,c-2\,b\right)+b^2+a^2}
\]
\end{eulerformula}
\begin{eulercomment}
Kita bisa menjadikannya sebagai fungsi Euler.
\end{eulercomment}
\begin{eulerprompt}
>function periradius(a,b,c) &= rabc;
\end{eulerprompt}
\begin{eulercomment}
Mari kita periksa hasilnya untuk poin A, B, C kita.
\end{eulercomment}
\begin{eulerprompt}
>a:=quadrance(B,C); b:=quadrance(A,C); c:=quadrance(A,B);
\end{eulerprompt}
\begin{eulercomment}
Radiusnya memang 1.
\end{eulercomment}
\begin{eulerprompt}
>periradius(a,b,c)
\end{eulerprompt}
\begin{euleroutput}
  1
\end{euleroutput}
\begin{eulercomment}
Faktanya, penyebaran CBA hanya bergantung pada b dan c. Ini adalah
teorema sudut akord.
\end{eulercomment}
\begin{eulerprompt}
>$spread(b,a,c)*rabc | ratsimp
\end{eulerprompt}
\begin{eulerformula}
\[
\frac{b}{4}
\]
\end{eulerformula}
\begin{eulercomment}
Sebenarnya sebarannya adalah b/(4r), dan kita melihat bahwa sudut akor
b adalah setengah dari sudut tengah.
\end{eulercomment}
\begin{eulerprompt}
>$doublespread(b/(4*r))-spread(b,r,r) | ratsimp
\end{eulerprompt}
\begin{eulerformula}
\[
0
\]
\end{eulerformula}
\begin{eulercomment}
\begin{eulercomment}
\eulerheading{Contoh 6: Jarak Minimal pada Bidang}
\begin{eulercomment}
\end{eulercomment}
\eulersubheading{Catatan awal}
\begin{eulercomment}
Fungsi yang, ke titik M di bidang, menetapkan jarak AM antara titik
tetap A dan M, memiliki garis level yang agak sederhana: lingkaran
berpusat di A.
\end{eulercomment}
\begin{eulerprompt}
>&remvalue();
>A=[-1,-1];
>function d1(x,y):=sqrt((x-A[1])^2+(y-A[2])^2)
>fcontour("d1",xmin=-2,xmax=0,ymin=-2,ymax=0,hue=1, ...
>title="If you see ellipses, please set your window square"):
\end{eulerprompt}
\eulerimg{27}{images/22305141017_Ardi Budi S_Geometri-141.png}
\begin{eulercomment}
dan grafiknya juga agak sederhana: bagian atas kerucut:
\end{eulercomment}
\begin{eulerprompt}
>plot3d("d1",xmin=-2,xmax=0,ymin=-2,ymax=0):
\end{eulerprompt}
\eulerimg{27}{images/22305141017_Ardi Budi S_Geometri-142.png}
\begin{eulercomment}
Tentu saja minimal 0 dicapai di A.

\end{eulercomment}
\eulersubheading{Dua titik}
\begin{eulercomment}
Sekarang kita melihat fungsi MA + MB dimana A dan B adalah dua titik
(tetap). Ini adalah "fakta yang terkenal" bahwa kurva level adalah
elips, titik fokusnya adalah A dan B; kecuali untuk minimum AB yang
konstan pada segmen [AB]:
\end{eulercomment}
\begin{eulerprompt}
>B=[1,-1];
>function d2(x,y):=d1(x,y)+sqrt((x-B[1])^2+(y-B[2])^2)
>fcontour("d2",xmin=-2,xmax=2,ymin=-3,ymax=1,hue=1):
\end{eulerprompt}
\eulerimg{27}{images/22305141017_Ardi Budi S_Geometri-143.png}
\begin{eulercomment}
Grafiknya lebih menarik:
\end{eulercomment}
\begin{eulerprompt}
>plot3d("d2",xmin=-2,xmax=2,ymin=-3,ymax=1):
\end{eulerprompt}
\eulerimg{27}{images/22305141017_Ardi Budi S_Geometri-144.png}
\begin{eulercomment}
Batasan ke baris (AB) lebih terkenal:
\end{eulercomment}
\begin{eulerprompt}
>plot2d("abs(x+1)+abs(x-1)",xmin=-3,xmax=3):
\end{eulerprompt}
\eulerimg{27}{images/22305141017_Ardi Budi S_Geometri-145.png}
\begin{eulercomment}
\end{eulercomment}
\eulersubheading{Tiga titik}
\begin{eulercomment}
Sekarang hal-hal menjadi kurang sederhana: Sedikit kurang diketahui
bahwa MA+MB+MC mencapai minimumnya pada satu titik bidang tetapi untuk
menentukannya kurang sederhana:

1) Jika salah satu sudut segitiga ABC lebih dari 120° (katakanlah
dalam A), maka minimum tercapai pada titik ini (katakanlah AB+AC).

Contoh:
\end{eulercomment}
\begin{eulerprompt}
>C=[-4,1];
>function d3(x,y):=d2(x,y)+sqrt((x-C[1])^2+(y-C[2])^2)
>plot3d("d3",xmin=-5,xmax=3,ymin=-4,ymax=4);
>insimg;
\end{eulerprompt}
\eulerimg{27}{images/22305141017_Ardi Budi S_Geometri-146.png}
\begin{eulerprompt}
>fcontour("d3",xmin=-4,xmax=1,ymin=-2,ymax=2,hue=1,title="The minimum is on A");
>P=(A_B_C_A)'; plot2d(P[1],P[2],add=1,color=12);
>insimg;
\end{eulerprompt}
\eulerimg{27}{images/22305141017_Ardi Budi S_Geometri-147.png}
\begin{eulercomment}
2) Tetapi jika semua sudut segitiga ABC kurang dari 120°, minimum
berada pada titik F di bagian dalam segitiga, yang merupakan
satu-satunya titik yang melihat sisi ABC dengan sudut yang sama (lalu
masing-masing 120°) :
\end{eulercomment}
\begin{eulerprompt}
>C=[-0.5,1];
>plot3d("d3",xmin=-2,xmax=2,ymin=-2,ymax=2):
\end{eulerprompt}
\eulerimg{27}{images/22305141017_Ardi Budi S_Geometri-148.png}
\begin{eulerprompt}
>fcontour("d3",xmin=-2,xmax=2,ymin=-2,ymax=2,hue=1,title="The Fermat point");
>P=(A_B_C_A)'; plot2d(P[1],P[2],add=1,color=12);
>insimg;
\end{eulerprompt}
\eulerimg{27}{images/22305141017_Ardi Budi S_Geometri-149.png}
\begin{eulercomment}
Merupakan kegiatan yang menarik untuk mewujudkan gambar di atas dengan
perangkat lunak geometri; sebagai contoh, saya tahu soft tertulis di
Java yang memiliki instruksi "garis kontur" ...

Semua ini di atas telah ditemukan oleh seorang hakim Prancis bernama
Pierre de Fermat; dia menulis surat kepada para penggila lainnya
seperti pendeta Marin Mersenne dan Blaise Pascal yang bekerja di
bagian pajak penghasilan. Jadi titik unik F sehingga FA + FB + FC
minimal disebut titik Fermat segitiga. Tetapi tampaknya beberapa tahun
sebelumnya, Torriccelli Italia telah menemukan titik ini sebelum
Fermat melakukannya! Pokoknya tradisinya adalah mencatat poin ini ...

\end{eulercomment}
\eulersubheading{Empat titik}
\begin{eulercomment}
Langkah selanjutnya adalah menambahkan titik D ke-4 dan mencoba
meminimalkan MA + MB + MC + MD; katakanlah bahwa Anda adalah operator
TV kabel dan ingin mencari di bidang mana Anda harus meletakkan antena
sehingga Anda dapat memberi makan empat desa dan menggunakan kabel
sesedikit mungkin!
\end{eulercomment}
\begin{eulerprompt}
>D=[1,1];
>function d4(x,y):=d3(x,y)+sqrt((x-D[1])^2+(y-D[2])^2)
>plot3d("d4",xmin=-1.5,xmax=1.5,ymin=-1.5,ymax=1.5):
\end{eulerprompt}
\eulerimg{27}{images/22305141017_Ardi Budi S_Geometri-150.png}
\begin{eulerprompt}
>fcontour("d4",xmin=-1.5,xmax=1.5,ymin=-1.5,ymax=1.5,hue=1);
>P=(A_B_C_D)'; plot2d(P[1],P[2],points=1,add=1,color=12);
>insimg;
\end{eulerprompt}
\eulerimg{27}{images/22305141017_Ardi Budi S_Geometri-151.png}
\begin{eulercomment}
Masih ada minimum dan tidak ada yang dicapai pada simpul A, B, C atau
D:
\end{eulercomment}
\begin{eulerprompt}
>function f(x):=d4(x[1],x[2])
>neldermin("f",[0.2,0.2])
\end{eulerprompt}
\begin{euleroutput}
  [0.142858,  0.142857]
\end{euleroutput}
\begin{eulercomment}
Tampaknya dalam kasus ini, koordinat titik optimal rasional atau
mendekati rasional ...

Sekarang ABCD adalah bujur sangkar, kami berharap bahwa titik optimal
adalah pusat ABCD:
\end{eulercomment}
\begin{eulerprompt}
>C=[-1,1];
>plot3d("d4",xmin=-1,xmax=1,ymin=-1,ymax=1):
\end{eulerprompt}
\eulerimg{27}{images/22305141017_Ardi Budi S_Geometri-152.png}
\begin{eulerprompt}
>fcontour("d4",xmin=-1.5,xmax=1.5,ymin=-1.5,ymax=1.5,hue=1);
>P=(A_B_C_D)'; plot2d(P[1],P[2],add=1,color=12,points=1);
>insimg;
\end{eulerprompt}
\eulerimg{27}{images/22305141017_Ardi Budi S_Geometri-153.png}
\eulerheading{Contoh 7: Bola Dandelin dengan Povray}
\begin{eulercomment}
Anda dapat menjalankan demonstrasi ini, jika Anda memiliki Povray
diinstal, dan pvengine.exe di jalur program.

Pertama kami menghitung jari-jari bola.

Jika Anda melihat gambar di bawah, Anda melihat bahwa kita membutuhkan
dua lingkaran yang menyentuh dua garis yang membentuk kerucut, dan
satu garis yang membentuk bidang yang memotong kerucut.

Kami menggunakan file geometry.e dari Euler untuk ini.
\end{eulercomment}
\begin{eulerprompt}
>load geometry;
\end{eulerprompt}
\begin{eulercomment}
Pertama, dua garis yang membentuk kerucut.
\end{eulercomment}
\begin{eulerprompt}
>g1 &= lineThrough([0,0],[1,a])
\end{eulerprompt}
\begin{euleroutput}
  
                               [- a, 1, 0]
  
\end{euleroutput}
\begin{eulerprompt}
>g2 &= lineThrough([0,0],[-1,a])
\end{eulerprompt}
\begin{euleroutput}
  
                              [- a, - 1, 0]
  
\end{euleroutput}
\begin{eulercomment}
Lalu baris ketiga.
\end{eulercomment}
\begin{eulerprompt}
>g &= lineThrough([-1,0],[1,1])
\end{eulerprompt}
\begin{euleroutput}
  
                               [- 1, 2, 1]
  
\end{euleroutput}
\begin{eulercomment}
Kita merencanakan semuanya sejauh ini.
\end{eulercomment}
\begin{eulerprompt}
>setPlotRange(-1,1,0,2);
>color(black); plotLine(g(),"")
>a:=2; color(blue); plotLine(g1(),""), plotLine(g2(),""):
\end{eulerprompt}
\eulerimg{27}{images/22305141017_Ardi Budi S_Geometri-154.png}
\begin{eulercomment}
Sekarang kita ambil titik umum pada sumbu y.
\end{eulercomment}
\begin{eulerprompt}
>P &= [0,u]
\end{eulerprompt}
\begin{euleroutput}
  
                                  [0, u]
  
\end{euleroutput}
\begin{eulercomment}
Hitung jarak ke g1.
\end{eulercomment}
\begin{eulerprompt}
>d1 &= distance(P,projectToLine(P,g1)); $d1
\end{eulerprompt}
\begin{eulerformula}
\[
\sqrt{\left(\frac{a^2\,u}{a^2+1}-u\right)^2+\frac{a^2\,u^2}{\left(a  ^2+1\right)^2}}
\]
\end{eulerformula}
\begin{eulercomment}
Hitung jarak ke g.
\end{eulercomment}
\begin{eulerprompt}
>d &= distance(P,projectToLine(P,g)); $d
\end{eulerprompt}
\begin{eulerformula}
\[
\sqrt{\left(\frac{u+2}{5}-u\right)^2+\frac{\left(2\,u-1\right)^2}{  25}}
\]
\end{eulerformula}
\begin{eulercomment}
Dan temukan pusat kedua lingkaran, di mana jaraknya sama.
\end{eulercomment}
\begin{eulerprompt}
>sol &= solve(d1^2=d^2,u); $sol
\end{eulerprompt}
\begin{eulerformula}
\[
\left[ u=\frac{-\sqrt{5}\,\sqrt{a^2+1}+2\,a^2+2}{4\,a^2-1} , u=  \frac{\sqrt{5}\,\sqrt{a^2+1}+2\,a^2+2}{4\,a^2-1} \right] 
\]
\end{eulerformula}
\begin{eulercomment}
Ada dua solusi.

Kami mengevaluasi solusi simbolis, dan menemukan kedua pusat, dan
kedua jarak.
\end{eulercomment}
\begin{eulerprompt}
>u := sol()
\end{eulerprompt}
\begin{euleroutput}
  [0.333333,  1]
\end{euleroutput}
\begin{eulerprompt}
>dd := d()
\end{eulerprompt}
\begin{euleroutput}
  [0.149071,  0.447214]
\end{euleroutput}
\begin{eulercomment}
Plot lingkaran ke dalam gambar.
\end{eulercomment}
\begin{eulerprompt}
>color(red);
>plotCircle(circleWithCenter([0,u[1]],dd[1]),"");
>plotCircle(circleWithCenter([0,u[2]],dd[2]),"");
>insimg;
\end{eulerprompt}
\eulerimg{27}{images/22305141017_Ardi Budi S_Geometri-158.png}
\eulersubheading{Plot dengan Povray}
\begin{eulercomment}
Selanjutnya kami merencanakan semuanya dengan Povray. Perhatikan bahwa
Anda mengubah perintah apa pun dalam urutan perintah Povray berikut,
dan menjalankan kembali semua perintah dengan Shift-Return.

Pertama kita memuat fungsi povray.
\end{eulercomment}
\begin{eulerprompt}
>load povray;
>defaultpovray="C:\(\backslash\)Program Files\(\backslash\)POV-Ray\(\backslash\)v3.7\(\backslash\)bin\(\backslash\)pvengine.exe"
\end{eulerprompt}
\begin{euleroutput}
  C:\(\backslash\)Program Files\(\backslash\)POV-Ray\(\backslash\)v3.7\(\backslash\)bin\(\backslash\)pvengine.exe
\end{euleroutput}
\begin{eulercomment}
Kita mengatur adegan dengan tepat.
\end{eulercomment}
\begin{eulerprompt}
>povstart(zoom=11,center=[0,0,0.5],height=10°,angle=140°);
\end{eulerprompt}
\begin{eulercomment}
Selanjutnya kita menulis dua bidang ke file Povray.
\end{eulercomment}
\begin{eulerprompt}
>writeln(povsphere([0,0,u[1]],dd[1],povlook(red)));
>writeln(povsphere([0,0,u[2]],dd[2],povlook(red)));
\end{eulerprompt}
\begin{eulercomment}
Dan kerucutnya, transparan.
\end{eulercomment}
\begin{eulerprompt}
>writeln(povcone([0,0,0],0,[0,0,a],1,povlook(lightgray,1)));
\end{eulerprompt}
\begin{eulercomment}
Kami menghasilkan pesawat terbatas pada kerucut.
\end{eulercomment}
\begin{eulerprompt}
>gp=g();
>pc=povcone([0,0,0],0,[0,0,a],1,"");
>vp=[gp[1],0,gp[2]]; dp=gp[3];
>writeln(povplane(vp,dp,povlook(blue,0.5),pc));
\end{eulerprompt}
\begin{eulercomment}
Sekarang kami menghasilkan dua titik pada lingkaran, di mana bola
menyentuh kerucut.
\end{eulercomment}
\begin{eulerprompt}
>function turnz(v) := return [-v[2],v[1],v[3]]
>P1=projectToLine([0,u[1]],g1()); P1=turnz([P1[1],0,P1[2]]);
>writeln(povpoint(P1,povlook(yellow)));
>P2=projectToLine([0,u[2]],g1()); P2=turnz([P2[1],0,P2[2]]);
>writeln(povpoint(P2,povlook(yellow)));
\end{eulerprompt}
\begin{eulercomment}
Kemudian kami menghasilkan dua titik di mana bola menyentuh bidang.
Ini adalah fokus elips.
\end{eulercomment}
\begin{eulerprompt}
>P3=projectToLine([0,u[1]],g()); P3=[P3[1],0,P3[2]];
>writeln(povpoint(P3,povlook(yellow)));
>P4=projectToLine([0,u[2]],g()); P4=[P4[1],0,P4[2]];
>writeln(povpoint(P4,povlook(yellow)));
\end{eulerprompt}
\begin{eulercomment}
Selanjutnya kita menghitung perpotongan P1P2 dengan bidang.
\end{eulercomment}
\begin{eulerprompt}
>t1=scalp(vp,P1)-dp; t2=scalp(vp,P2)-dp; P5=P1+t1/(t1-t2)*(P2-P1);
>writeln(povpoint(P5,povlook(yellow)));
\end{eulerprompt}
\begin{eulercomment}
Kami menghubungkan titik dengan segmen garis.
\end{eulercomment}
\begin{eulerprompt}
>writeln(povsegment(P1,P2,povlook(yellow)));
>writeln(povsegment(P5,P3,povlook(yellow)));
>writeln(povsegment(P5,P4,povlook(yellow)));
\end{eulerprompt}
\begin{eulercomment}
Sekarang kami membuat pita abu-abu, di mana bola menyentuh kerucut.
\end{eulercomment}
\begin{eulerprompt}
>pcw=povcone([0,0,0],0,[0,0,a],1.01);
>pc1=povcylinder([0,0,P1[3]-defaultpointsize/2],[0,0,P1[3]+defaultpointsize/2],1);
>writeln(povintersection([pcw,pc1],povlook(gray)));
>pc2=povcylinder([0,0,P2[3]-defaultpointsize/2],[0,0,P2[3]+defaultpointsize/2],1);
>writeln(povintersection([pcw,pc2],povlook(gray)));
\end{eulerprompt}
\begin{eulercomment}
Mulai program Povray.
\end{eulercomment}
\begin{eulerprompt}
>povend();
\end{eulerprompt}
\eulerimg{27}{images/22305141017_Ardi Budi S_Geometri-159.png}
\begin{eulercomment}
Untuk mendapatkan Anaglyph ini, kita perlu memasukkan semuanya ke
dalam fungsi scene. Fungsi ini akan digunakan dua kali nanti.
\end{eulercomment}
\begin{eulerprompt}
>function scene () ...
\end{eulerprompt}
\begin{eulerudf}
  global a,u,dd,g,g1,defaultpointsize;
  writeln(povsphere([0,0,u[1]],dd[1],povlook(red)));
  writeln(povsphere([0,0,u[2]],dd[2],povlook(red)));
  writeln(povcone([0,0,0],0,[0,0,a],1,povlook(lightgray,1)));
  gp=g();
  pc=povcone([0,0,0],0,[0,0,a],1,"");
  vp=[gp[1],0,gp[2]]; dp=gp[3];
  writeln(povplane(vp,dp,povlook(blue,0.5),pc));
  P1=projectToLine([0,u[1]],g1()); P1=turnz([P1[1],0,P1[2]]);
  writeln(povpoint(P1,povlook(yellow)));
  P2=projectToLine([0,u[2]],g1()); P2=turnz([P2[1],0,P2[2]]);
  writeln(povpoint(P2,povlook(yellow)));
  P3=projectToLine([0,u[1]],g()); P3=[P3[1],0,P3[2]];
  writeln(povpoint(P3,povlook(yellow)));
  P4=projectToLine([0,u[2]],g()); P4=[P4[1],0,P4[2]];
  writeln(povpoint(P4,povlook(yellow)));
  t1=scalp(vp,P1)-dp; t2=scalp(vp,P2)-dp; P5=P1+t1/(t1-t2)*(P2-P1);
  writeln(povpoint(P5,povlook(yellow)));
  writeln(povsegment(P1,P2,povlook(yellow)));
  writeln(povsegment(P5,P3,povlook(yellow)));
  writeln(povsegment(P5,P4,povlook(yellow)));
  pcw=povcone([0,0,0],0,[0,0,a],1.01);
  pc1=povcylinder([0,0,P1[3]-defaultpointsize/2],[0,0,P1[3]+defaultpointsize/2],1);
  writeln(povintersection([pcw,pc1],povlook(gray)));
  pc2=povcylinder([0,0,P2[3]-defaultpointsize/2],[0,0,P2[3]+defaultpointsize/2],1);
  writeln(povintersection([pcw,pc2],povlook(gray)));
  endfunction
\end{eulerudf}
\begin{eulercomment}
Anda membutuhkan kacamata merah / cyan untuk mengapresiasi efek
berikut.
\end{eulercomment}
\begin{eulerprompt}
>povanaglyph("scene",zoom=11,center=[0,0,0.5],height=10°,angle=140°);
\end{eulerprompt}
\eulerimg{27}{images/22305141017_Ardi Budi S_Geometri-160.png}
\eulerheading{Contoh 8: Geometri Bumi}
\begin{eulercomment}
Di notebook ini, kami ingin melakukan beberapa komputasi bola.
Fungsi-fungsi tersebut terdapat dalam file "spherical.e" di folder
contoh. Kita perlu memuat file itu dulu.
\end{eulercomment}
\begin{eulerprompt}
>load spherical.e
\end{eulerprompt}
\begin{euleroutput}
  Spherical functions for Euler. 
\end{euleroutput}
\begin{eulercomment}
Untuk memasukkan posisi geografis, kami menggunakan vektor dengan dua
koordinat dalam radian (utara dan timur, nilai negatif untuk selatan
dan barat). Berikut koordinat Kampus FMIPA UNY.
\end{eulercomment}
\begin{eulerprompt}
>FMIPA=[rad(-7,-46.467),rad(110,23.05)]
\end{eulerprompt}
\begin{euleroutput}
  [-0.13569,  1.92657]
\end{euleroutput}
\begin{eulercomment}
Anda dapat mencetak posisi ini dengan sposprint (cetak posisi bola).
\end{eulercomment}
\begin{eulerprompt}
>sposprint(FMIPA) // posisi garis lintang dan garis bujur FMIPA UNY
\end{eulerprompt}
\begin{euleroutput}
  S 7°46.467' E 110°23.050'
\end{euleroutput}
\begin{eulercomment}
Mari kita tambahkan dua kota lagi, Solo dan Semarang.
\end{eulercomment}
\begin{eulerprompt}
>Solo=[rad(-7,-34.333),rad(110,49.683)]; Semarang=[rad(-6,-59.05),rad(110,24.533)];
>sposprint(Solo), sposprint(Semarang),
\end{eulerprompt}
\begin{euleroutput}
  S 7°34.333' E 110°49.683'
  S 6°59.050' E 110°24.533'
\end{euleroutput}
\begin{eulercomment}
Pertama kita menghitung vektor dari satu bola ke bola lainnya pada
bola ideal. Vektor ini adalah [heading, distance] dalam radian. Untuk
menghitung jarak di bumi, kita mengalikan dengan jari-jari bumi pada
garis lintang 7°.
\end{eulercomment}
\begin{eulerprompt}
>br=svector(FMIPA,Solo); degprint(br[1]), br[2]*rearth(7°)->km // perkiraan jarak FMIPA-Solo
\end{eulerprompt}
\begin{euleroutput}
  65°20'26.60''
  53.8945384608
\end{euleroutput}
\begin{eulercomment}
Ini adalah perkiraan yang bagus. Rutinitas berikut menggunakan
perkiraan yang lebih baik. Pada jarak yang begitu dekat hasilnya
hampir sama.
\end{eulercomment}
\begin{eulerprompt}
>esdist(FMIPA,Semarang)->" km", // perkiraan jarak FMIPA-Semarang
\end{eulerprompt}
\begin{euleroutput}
  88.0114026318 km
\end{euleroutput}
\begin{eulercomment}
Ada fungsi untuk heading, dengan mempertimbangkan bentuk bumi yang
elips. Sekali lagi, kami mencetak dengan cara yang canggih.
\end{eulercomment}
\begin{eulerprompt}
>sdegprint(esdir(FMIPA,Solo))
\end{eulerprompt}
\begin{euleroutput}
       65.34°
\end{euleroutput}
\begin{eulercomment}
Sudut segitiga melebihi 180° pada bola.
\end{eulercomment}
\begin{eulerprompt}
>asum=sangle(Solo,FMIPA,Semarang)+sangle(FMIPA,Solo,Semarang)+sangle(FMIPA,Semarang,Solo); degprint(asum)
\end{eulerprompt}
\begin{euleroutput}
  180°0'10.77''
\end{euleroutput}
\begin{eulercomment}
Ini dapat digunakan untuk menghitung luas segitiga. Catatan: Untuk
segitiga kecil, ini tidak akurat karena kesalahan pengurangan dalam
asum-pi.
\end{eulercomment}
\begin{eulerprompt}
>(asum-pi)*rearth(48°)^2->" km^2", //perkiraan luas segitiga FMIPA-Solo-Semarang
\end{eulerprompt}
\begin{euleroutput}
  2116.02948749 km^2
\end{euleroutput}
\begin{eulercomment}
Ada fungsi untuk ini, yang menggunakan garis lintang rata-rata
segitiga untuk menghitung jari-jari bumi, dan menangani kesalahan
pembulatan untuk segitiga yang sangat kecil.
\end{eulercomment}
\begin{eulerprompt}
>esarea(Solo,FMIPA,Semarang)->" km^2", //perkiraan yang sama dengan fungsi esarea()
\end{eulerprompt}
\begin{euleroutput}
  2123.64310526 km^2
\end{euleroutput}
\begin{eulercomment}
Kami juga dapat menambahkan vektor ke posisi. Vektor berisi heading
dan jarak, keduanya dalam radian. Untuk mendapatkan vektor, kami
menggunakan svector. Untuk menambahkan vektor ke posisi, kami
menggunakan saddvector.
\end{eulercomment}
\begin{eulerprompt}
>v=svector(FMIPA,Solo); sposprint(saddvector(FMIPA,v)), sposprint(Solo),
\end{eulerprompt}
\begin{euleroutput}
  S 7°34.333' E 110°49.683'
  S 7°34.333' E 110°49.683'
\end{euleroutput}
\begin{eulercomment}
Fungsi-fungsi ini mengasumsikan bola yang ideal. Hal yang sama di
bumi.
\end{eulercomment}
\begin{eulerprompt}
>sposprint(esadd(FMIPA,esdir(FMIPA,Solo),esdist(FMIPA,Solo))), sposprint(Solo),
\end{eulerprompt}
\begin{euleroutput}
  S 7°34.333' E 110°49.683'
  S 7°34.333' E 110°49.683'
\end{euleroutput}
\begin{eulercomment}
Mari kita beralih ke contoh yang lebih besar, Tugu Jogja dan Monas
Jakarta (menggunakan Google Earth untuk mencari koordinatnya).
\end{eulercomment}
\begin{eulerprompt}
>Tugu=[-7.7833°,110.3661°]; Monas=[-6.175°,106.811944°];
>sposprint(Tugu), sposprint(Monas)
\end{eulerprompt}
\begin{euleroutput}
  S 7°46.998' E 110°21.966'
  S 6°10.500' E 106°48.717'
\end{euleroutput}
\begin{eulercomment}
Menurut Google Earth, jaraknya 429,66 km. Kami mendapatkan perkiraan
yang bagus.
\end{eulercomment}
\begin{eulerprompt}
>esdist(Tugu,Monas)->" km", // perkiraan jarak Tugu Jogja - Monas Jakarta
\end{eulerprompt}
\begin{euleroutput}
  431.565659488 km
\end{euleroutput}
\begin{eulercomment}
Judulnya sama dengan yang dihitung di Google Earth.
\end{eulercomment}
\begin{eulerprompt}
>degprint(esdir(Tugu,Monas))
\end{eulerprompt}
\begin{euleroutput}
  294°17'2.85''
\end{euleroutput}
\begin{eulercomment}
Namun, kita tidak lagi mendapatkan posisi target yang tepat, jika kita
menambahkan heading dan jarak ke posisi semula. Hal ini terjadi,
karena kita tidak menghitung fungsi invers secara tepat, tetapi
mengambil perkiraan jari-jari bumi di sepanjang jalan.
\end{eulercomment}
\begin{eulerprompt}
>sposprint(esadd(Tugu,esdir(Tugu,Monas),esdist(Tugu,Monas)))
\end{eulerprompt}
\begin{euleroutput}
  S 6°10.500' E 106°48.717'
\end{euleroutput}
\begin{eulercomment}
Namun, kesalahannya tidak besar.
\end{eulercomment}
\begin{eulerprompt}
>sposprint(Monas),
\end{eulerprompt}
\begin{euleroutput}
  S 6°10.500' E 106°48.717'
\end{euleroutput}
\begin{eulercomment}
Tentunya kita tidak bisa berlayar dengan tujuan yang sama dari satu
tujuan ke tujuan lainnya, jika kita ingin mengambil jalur terpendek.
Bayangkan, Anda terbang NE mulai dari titik mana pun di bumi. Kemudian
Anda akan berputar ke kutub utara. Lingkaran besar tidak mengikuti
arah yang konstan!

Perhitungan berikut menunjukkan bahwa kami jauh dari tujuan yang
benar, jika kami menggunakan tajuk yang sama selama perjalanan kami.
\end{eulercomment}
\begin{eulerprompt}
>dist=esdist(Tugu,Monas); hd=esdir(Tugu,Monas);
\end{eulerprompt}
\begin{eulercomment}
Sekarang kita tambahkan 10 kali sepersepuluh jaraknya, menggunakan
heading ke Monas, kita sampai di Tugu.
\end{eulercomment}
\begin{eulerprompt}
>p=Tugu; loop 1 to 10; p=esadd(p,hd,dist/10); end;
\end{eulerprompt}
\begin{eulercomment}
Hasilnya masih jauh.
\end{eulercomment}
\begin{eulerprompt}
>sposprint(p), skmprint(esdist(p,Monas))
\end{eulerprompt}
\begin{euleroutput}
  S 6°11.250' E 106°48.372'
       1.529km
\end{euleroutput}
\begin{eulercomment}
Sebagai contoh lain, mari kita ambil dua titik di bumi pada ketinggian
yang sama.
\end{eulercomment}
\begin{eulerprompt}
> P1=[30°,10°]; P2=[30°,50°];
\end{eulerprompt}
\begin{eulercomment}
Jalur terpendek dari P1 ke P2 bukanlah lingkaran dengan garis lintang
30°, tetapi jalur yang lebih pendek mulai 10° lebih jauh ke utara di
P1.
\end{eulercomment}
\begin{eulerprompt}
>sdegprint(esdir(P1,P2))
\end{eulerprompt}
\begin{euleroutput}
       79.69°
\end{euleroutput}
\begin{eulercomment}
Tapi, jika kita mengikuti pembacaan kompas ini, kita akan berputar ke
kutub utara! Jadi kita harus menyesuaikan arah tujuan kita di
sepanjang jalan. Untuk tujuan kasar, kami menyesuaikannya pada 1/10
dari jarak total.
\end{eulercomment}
\begin{eulerprompt}
>p=P1;  dist=esdist(P1,P2); ...
>  loop 1 to 10; dir=esdir(p,P2); sdegprint(dir), p=esadd(p,dir,dist/10); end;
\end{eulerprompt}
\begin{euleroutput}
       79.69°
       81.67°
       83.71°
       85.78°
       87.89°
       90.00°
       92.12°
       94.22°
       96.29°
       98.33°
\end{euleroutput}
\begin{eulercomment}
Jaraknya tidak tepat, karena kita akan menambahkan sedikit kesalahan,
jika kita mengikuti tajuk yang sama terlalu lama.
\end{eulercomment}
\begin{eulerprompt}
>skmprint(esdist(p,P2))
\end{eulerprompt}
\begin{euleroutput}
       0.203km
\end{euleroutput}
\begin{eulercomment}
Kami mendapatkan perkiraan yang baik, jika kami menyesuaikan heading
setelah setiap 1/100 dari total jarak dari Tugu ke Monas.
\end{eulercomment}
\begin{eulerprompt}
>p=Tugu; dist=esdist(Tugu,Monas); ...
>  loop 1 to 100; p=esadd(p,esdir(p,Monas),dist/100); end;
>skmprint(esdist(p,Monas))
\end{eulerprompt}
\begin{euleroutput}
       0.000km
\end{euleroutput}
\begin{eulercomment}
Untuk keperluan navigasi, kita bisa mendapatkan urutan posisi GPS di
sepanjang lingkaran besar menuju Monas dengan fungsi navigasi.
\end{eulercomment}
\begin{eulerprompt}
>load spherical; v=navigate(Tugu,Monas,10); ...
>  loop 1 to rows(v); sposprint(v[#]), end;
\end{eulerprompt}
\begin{euleroutput}
  S 7°46.998' E 110°21.966'
  S 7°37.422' E 110°0.573'
  S 7°27.829' E 109°39.196'
  S 7°18.219' E 109°17.834'
  S 7°8.592' E 108°56.488'
  S 6°58.948' E 108°35.157'
  S 6°49.289' E 108°13.841'
  S 6°39.614' E 107°52.539'
  S 6°29.924' E 107°31.251'
  S 6°20.219' E 107°9.977'
  S 6°10.500' E 106°48.717'
\end{euleroutput}
\begin{eulercomment}
Kami menulis sebuah fungsi, yang menggambarkan bumi, dua posisi, dan
posisi di antaranya.
\end{eulercomment}
\begin{eulerprompt}
>function testplot ...
\end{eulerprompt}
\begin{eulerudf}
  useglobal;
  plotearth;
  plotpos(Tugu,"Tugu Jogja"); plotpos(Monas,"Tugu Monas");
  plotposline(v);
  endfunction
\end{eulerudf}
\begin{eulercomment}
Sekarang plot semuanya.
\end{eulercomment}
\begin{eulerprompt}
>plot3d("testplot",angle=25, height=6,>own,>user,zoom=4):
\end{eulerprompt}
\eulerimg{27}{images/22305141017_Ardi Budi S_Geometri-161.png}
\begin{eulercomment}
Atau gunakan plot3d untuk mendapatkan tampilan anaglyphnya. Ini
terlihat sangat bagus dengan kacamata merah / cyan.
\end{eulercomment}
\begin{eulerprompt}
>plot3d("testplot",angle=25,height=6,distance=5,own=1,anaglyph=1,zoom=4):
\end{eulerprompt}
\eulerimg{27}{images/22305141017_Ardi Budi S_Geometri-162.png}
\eulerheading{Latihan}
\begin{eulercomment}
1. Gambarlah segi-n beraturan jika diketahui titik pusat O, n, dan
jarak titik pusat ke titik-titik sudut segi-n tersebut (jari-jari
lingkaran luar segi-n), r.

Petunjuk:

- Besar sudut pusat yang menghadap masing-masing sisi segi-n adalah
(360/n).\\
- Titik-titik sudut segi-n merupakan perpotongan lingkaran luar segi-n
dan garis-garis yang melalui pusat dan saling membentuk sudut sebesar
kelipatan (360/n).\\
- Untuk n ganjil, pilih salah satu titik sudut adalah di atas.\\
- Untuk n genap, pilih 2 titik di kanan dan kiri lurus dengan titik
pusat.\\
- Anda dapat menggambar segi-3, 4, 5, 6, 7, dst beraturan.

2. Gambarlah suatu parabola yang melalui 3 titik yang diketahui.

Petunjuk:\\
- Misalkan persamaan parabolanya y= ax\textasciicircum{}2+bx+c.\\
- Substitusikan koordinat titik-titik yang diketahui ke persamaan
tersebut.\\
- Selesaikan SPL yang terbentuk untuk mendapatkan nilai-nilai a, b, c.

3. Gambarlah suatu segi-4 yang diketahui keempat titik sudutnya,
misalnya A, B, C, D.\\
\end{eulercomment}
\begin{eulerttcomment}
   - Tentukan apakah segi-4 tersebut merupakan segi-4 garis singgung
\end{eulerttcomment}
\begin{eulercomment}
(sisinya-sisintya merupakan garis singgung lingkaran yang sama yakni
lingkaran dalam segi-4 tersebut).\\
\end{eulercomment}
\begin{eulerttcomment}
   - Suatu segi-4 merupakan segi-4 garis singgung apabila keempat
\end{eulerttcomment}
\begin{eulercomment}
garis bagi sudutnya bertemu di satu titik.\\
\end{eulercomment}
\begin{eulerttcomment}
   - Jika segi-4 tersebut merupakan segi-4 garis singgung, gambar
\end{eulerttcomment}
\begin{eulercomment}
lingkaran dalamnya.\\
\end{eulercomment}
\begin{eulerttcomment}
   - Tunjukkan bahwa syarat suatu segi-4 merupakan segi-4 garis
\end{eulerttcomment}
\begin{eulercomment}
singgung apabila hasil kali panjang sisi-sisi yang berhadapan sama.

4. Gambarlah suatu ellips jika diketahui kedua titik fokusnya,
misalnya P dan Q. Ingat ellips dengan fokus P dan Q adalah tempat
kedudukan titik-titik yang jumlah jarak ke P dan ke Q selalu sama
(konstan).

5. Gambarlah suatu hiperbola jika diketahui kedua titik fokusnya,
misalnya P dan Q. Ingat ellips dengan fokus P dan Q adalah tempat
kedudukan titik-titik yang selisih jarak ke P dan ke Q selalu sama
(konstan).

\end{eulercomment}
\eulersubheading{Jawab }
\begin{eulercomment}
1. Gambarlah segi-n beraturan jika diketahui titik pusat O, n, dan
jarak titik pusat ke titik-titik sudut segi-n tersebut\\
(jari-jari lingkaran luar segi-n), r.

a. segi-3
\end{eulercomment}
\begin{eulerprompt}
>setPlotRange(25);
>v = [1,1];
>O = [0,0];
>color(1); plotPoint(O, "O");
>circ = circleWithCenter(O, 20);
>line1 = lineWithDirection(O, turn(v, pi/4));
>line2 = lineWithDirection(O, turn(v,(11pi)/12));
>line3 = lineWithDirection(O, turn(v, 19pi/12));
>intersect1 = lineCircleIntersections(line1, circ);
>intersect2 = lineCircleIntersections(line2, circ);
>intersect3 = lineCircleIntersections(line3, circ);
>color(4); plotSegment(intersect1, intersect2, "a");
>color(4); plotSegment(intersect2, intersect3, "b");
>color(4); plotSegment(intersect3, intersect1, "c"):
\end{eulerprompt}
\eulerimg{27}{images/22305141017_Ardi Budi S_Geometri-163.png}
\begin{eulercomment}
b. segi-4
\end{eulercomment}
\begin{eulerprompt}
>setPlotRange(25);
>v = [1, 1];
>O = [0, 0];
>color(1); plotPoint(O, "O");
>circ = circleWithCenter(O, 20);
>line1 = lineWithDirection(O, turn(v, -pi/4));
>line2 = lineWithDirection(O, turn(v, pi/4));
>line3 = lineWithDirection(O, turn(v, 3pi/4));
>line4 = lineWithDirection(O, turn(v, 5pi/4));
>intersect1 = lineCircleIntersections(line1, circ);
>intersect2 = lineCircleIntersections(line2, circ);
>intersect3 = lineCircleIntersections(line3, circ);
>intersect4 = lineCircleIntersections(line4, circ);
>color(4); plotSegment(intersect1, intersect2, "a");
>color(4); plotSegment(intersect2, intersect3, "b");
>color(4); plotSegment(intersect3, intersect4, "c");
>color(4); plotSegment(intersect4, intersect1, "d"):
\end{eulerprompt}
\eulerimg{27}{images/22305141017_Ardi Budi S_Geometri-164.png}
\begin{eulercomment}
c. segi-5
\end{eulercomment}
\begin{eulerprompt}
>setPlotRange(25);
>v = [1, 1];
>O = [0, 0];
>color(1); plotPoint(O, "O");
>circ = circleWithCenter(O, 20);
>line1 = lineWithDirection(O, turn(v, pi/4));
>line2 = lineWithDirection(O, turn(v, 13pi/20));
>line3 = lineWithDirection(O, turn(v, 21pi/20));
>line4 = lineWithDirection(O, turn(v, 29pi/20));
>line5 = lineWithDirection(O, turn(v, 37pi/20));
>intersect1 = lineCircleIntersections(line1, circ);
>intersect2 = lineCircleIntersections(line2, circ);
>intersect3 = lineCircleIntersections(line3, circ);
>intersect4 = lineCircleIntersections(line4, circ);
>intersect5 = lineCircleIntersections(line5, circ);
>color(4); plotSegment(intersect1, intersect2, "a");
>color(4); plotSegment(intersect2, intersect3, "b");
>color(4); plotSegment(intersect3, intersect4, "c");
>color(4); plotSegment(intersect4, intersect5, "d");
>color(4); plotSegment(intersect5, intersect1, "e"):
\end{eulerprompt}
\eulerimg{27}{images/22305141017_Ardi Budi S_Geometri-165.png}
\begin{eulercomment}
d. segi-6
\end{eulercomment}
\begin{eulerprompt}
>setPlotRange(25);
>v = [1, 1];
>O = [0, 0];
>color(1); plotPoint(O, "O");
>circ = circleWithCenter(O, 20);
>line1 = lineWithDirection(O, turn(v, -pi/4));
>line2 = lineWithDirection(O, turn(v, pi/12));
>line3 = lineWithDirection(O, turn(v, 5pi/12));
>line4 = lineWithDirection(O, turn(v, 3pi/4));
>line5 = lineWithDirection(O, turn(v, 13pi/12));
>line6 = lineWithDirection(O, turn(v, 17pi/12));
>intersect1 = lineCircleIntersections(line1, circ);
>intersect2 = lineCircleIntersections(line2, circ);
>intersect3 = lineCircleIntersections(line3, circ);
>intersect4 = lineCircleIntersections(line4, circ);
>intersect5 = lineCircleIntersections(line5, circ);
>intersect6 = lineCircleIntersections(line6, circ);
>color(4); plotSegment(intersect1, intersect2, "a");
>color(4); plotSegment(intersect2, intersect3, "b");
>color(4); plotSegment(intersect3, intersect4, "c");
>color(4); plotSegment(intersect4, intersect5, "d");
>color(4); plotSegment(intersect5, intersect6, "e");
>color(4); plotSegment(intersect6, intersect1, "f"):
\end{eulerprompt}
\eulerimg{27}{images/22305141017_Ardi Budi S_Geometri-166.png}
\eulersubheading{}
\begin{eulercomment}
2. Menggambar Parabola melalui 3 titik yang diketahui
\end{eulercomment}
\begin{eulerprompt}
>setPlotRange(-11,15,-8,22);
>A = [-1, 6];
>B = [2, 0];
>C = [4, 6];
>color(1); plotPoint(A, "A");
>color(1); plotPoint(B, "B");
>color(1); plotPoint(C, "C"):
\end{eulerprompt}
\eulerimg{27}{images/22305141017_Ardi Budi S_Geometri-167.png}
\begin{eulerprompt}
>sol &= solve([a-b+c=6, 4*a+2*b+c=0, 16*a+4*b+c=6], [a,b,c]); $sol
\end{eulerprompt}
\begin{eulerformula}
\[
\left[ \left[ a=1 , b=-3 , c=2 \right]  \right] 
\]
\end{eulerformula}
\begin{eulerprompt}
>y &= x^2 - 3*x + 2; $y
\end{eulerprompt}
\begin{eulerformula}
\[
x^2-3\,x+2
\]
\end{eulerformula}
\begin{eulerprompt}
>plot2d(&y,add=1,color=olive,thickness=2);
>plotLabel("y = x^2 - 3x + 2", [6,5]):
\end{eulerprompt}
\eulerimg{27}{images/22305141017_Ardi Budi S_Geometri-170.png}
\begin{eulercomment}
\end{eulercomment}
\eulersubheading{}
\begin{eulercomment}
3. Gambar persegi yang diketahui 4 titiknya
\end{eulercomment}
\begin{eulerprompt}
>setPlotRange(30);
>O = [0, 0];
>color(1); plotPoint(O, "O");
>A = [15, 15];
>B = [-15, 15];
>C = [-15, -15];
>D = [15, -15];
>color(1); plotPoint(A, "A");
>color(1); plotPoint(B, "B");
>color(1); plotPoint(C, "C");
>color(1); plotPoint(D, "D");
>color(1); plotSegment(A, B, "");
>color(1); plotSegment(B, C, "");
>color(1); plotSegment(C, D, "");
>color(1); plotSegment(D, A, ""):
\end{eulerprompt}
\eulerimg{27}{images/22305141017_Ardi Budi S_Geometri-171.png}
\begin{eulercomment}
a. Akan ditentukan bahwa segi-4 sisi-sisinya merupakan garis singgung
lingkaran
\end{eulercomment}
\begin{eulerprompt}
>circIn = circleWithCenter(O, 15);
>color(4); plotCircle(circIn, ""):
\end{eulerprompt}
\eulerimg{27}{images/22305141017_Ardi Budi S_Geometri-172.png}
\begin{eulercomment}
Dari grafik atau gambar terlihat bahwa keempat sisi dari segi-4 adalah
garis singgung dari lingkaran dalam segi-4 tersebut.

b. Suatu segi-4 merupakan segi-4 garis singgung apabila keempat garis
bagi sudutnya bertemu di satu titik.
\end{eulercomment}
\begin{eulerprompt}
>color(3); plotSegment(A, C, "");
>color(3); plotSegment(B, D, ""):
\end{eulerprompt}
\eulerimg{27}{images/22305141017_Ardi Budi S_Geometri-173.png}
\begin{eulercomment}
Dari grafik atau gambar diatas terlihat bahwa pada segi-4 garis
singgung, keempat garis bagi sudutnya bertemu di satu titk yaitu\\
titik O(0,0).

c. Jika segi-4 tersebut merupakan segi-4 garis singgung, gambar
lingkaran dalamnya.
\end{eulercomment}
\begin{eulerprompt}
>setPlotRange(30);
>O = [0, 0];
>color(1); plotPoint(O, "O");
>A = [15, 15];
>B = [-15, 15];
>C = [-15, -15];
>D = [15, -15];
>color(1); plotPoint(A, "A");
>color(1); plotPoint(B, "B");
>color(1); plotPoint(C, "C");
>color(1); plotPoint(D, "D");
>color(1); plotSegment(A, B, "");
>color(1); plotSegment(B, C, "");
>color(1); plotSegment(C, D, "");
>color(1); plotSegment(D, A, "");
>circIn = circleWithCenter(O, 15);
>color(4); plotCircle(circIn, "lingkaran dalam"):
\end{eulerprompt}
\eulerimg{27}{images/22305141017_Ardi Budi S_Geometri-174.png}
\begin{eulercomment}
d. Menunjukkan bahwa syarat suatu segi-4 merupakan segi-4 garis
singgung apabila hasil kali panjang sisi-sisi yang\\
berhadapan sama.

- sisi AB dengan sisi CD
\end{eulercomment}
\begin{eulerprompt}
>AB = distance(A, B);
>CD = distance(C, D);
>AB * CD
\end{eulerprompt}
\begin{euleroutput}
  900
\end{euleroutput}
\begin{eulercomment}
sisi AD dengan sisi BC
\end{eulercomment}
\begin{eulerprompt}
>AD = distance(A, D);
>BC = distance(B, C);
>AD * BC
\end{eulerprompt}
\begin{euleroutput}
  900
\end{euleroutput}
\begin{eulercomment}
Didapatkan bahwa AB*CD = AD*BC atau hasil kali panjang sisi-sisi yang
berhadapan sama.
\end{eulercomment}
\eulersubheading{}
\begin{eulercomment}
4. Menggambar Ellips dengan titik fokus F1(-3,0) dan F2(3,0)
\end{eulercomment}
\begin{eulerprompt}
>setPlotRange(9);
>color(4);
>O = [0, 0]; plotPoint(O);
>F1 = [-3, 0]; plotPoint(F1);
>F2 = [3, 0]; plotPoint(F2);
>A = [-5, 0]; plotPoint(A);
>B = [0, 4]; plotPoint(B);
>C = [5, 0]; plotPoint(C);
>D = [0, -4]; plotPoint(D);
>color(1);
>sbMayor = lineThrough(A, C); plotLine(sbMayor, ""); plotLabel("Sumbu Mayor", [7, 0.1]);
>sbMinor = lineThrough(B, D); plotLine(sbMinor, ""); plotLabel("Sumbu Minor", [1.5, -7]);
>color(4);
>plotSegment(F1, B, "a");
>plotSegment(F2, B, "a");
>color(3);
>u = linspace(-2pi,2pi,200); ...
>x = 5*sin(u); y = 4*cos(u); ...
>aspect(1); plot2d(x, y, >add):
\end{eulerprompt}
\eulerimg{27}{images/22305141017_Ardi Budi S_Geometri-175.png}
\begin{eulercomment}
\end{eulercomment}
\eulersubheading{}
\begin{eulercomment}
5. Menggambar hiperbola dengan titik fokus yaitu di F1(0,-2) dan
F2(0,2)
\end{eulercomment}
\begin{eulerprompt}
>setPlotRange(9);
>color(3);
>O = [0, 0]; plotPoint(O);
>F1 = [0, -3]; plotPoint(F1);
>F2 = [0, 3]; plotPoint(F2);
>A = [0, -2]; plotPoint(A);
>B = [0, 2]; plotPoint(B);
>color(1);
>sbMayor = lineThrough(B, D); plotLine(sbMayor, ""); plotLabel("Sumbu Mayor", [1.5, -7]);
>sbMinor = lineWithDirection(O, [1, 0]); plotLine(sbMinor, ""); plotLabel("Sumbu Minor", [7, 0.1]);
>color(4);
>u = linspace(-2pi,2pi,200); ...
>y = 2*sec(u); x = sqrt(5)*tan(u); ...
>aspect(1); plot2d(x, y, >add):
\end{eulerprompt}
\eulerimg{27}{images/22305141017_Ardi Budi S_Geometri-176.png}
\begin{eulercomment}
\begin{eulercomment}
\eulerheading{Beberapa soal/masalah yang menarik dari buku}
\begin{eulercomment}
1. Diberikan dua garis yang saling berpotongan yaitu garis AB dan CD.
Garis AB melalui titik A(-3,4) dan titik B(4,0), sedangkan garis CD
melalui titik C(-1,0) dan titik D(1,5). Tentukan koordinat titik
potong pada kedua garis tersebut!
\end{eulercomment}
\begin{eulerprompt}
>setPlotRange(-6,6,-6,6); // menggambar bidang kartesius
\end{eulerprompt}
\begin{eulercomment}
Mendefinisikan setiap titik koodinat yang diketahui:\\
garis AB: titik A(-3,4) dan titik B(4,0)\\
garis CD: titik C(-1,0) dan titik D(1,5)
\end{eulercomment}
\begin{eulerprompt}
>A=[-3,4]; plotPoint(A,"A"); 
>B=[4,0]; plotPoint(B,"B");
>C=[-1,0]; plotPoint(C,"C");
>D=[1,5]; plotPoint(D,"D");
\end{eulerprompt}
\begin{eulercomment}
Mendefinisikan dan Menggambar Garis AB dan CD
\end{eulercomment}
\begin{eulerprompt}
>plotSegment(A,B,"c"); // c=AB menggambar garis c dan b
>plotSegment(C,D,"b"); // b=CD
>lineThrough(A,B); // garis yang melalui titik A dan titik B
>lineThrough(C,D); // garis yang melalui titik C dan titik D
\end{eulerprompt}
\begin{eulercomment}
Menentukan titik potong kedua garis:\\
Misalkan:\\
Titik potong garis AB dan garis CD = P
\end{eulercomment}
\begin{eulerprompt}
>P=lineIntersection(lineThrough(A,B),lineThrough(C,D));
>plotPoint(P,value=1):
\end{eulerprompt}
\eulerimg{27}{images/22305141017_Ardi Budi S_Geometri-177.png}
\begin{eulercomment}
=\textgreater{} Diperoleh titik P(-0.07,2.3)\\
Maka, koordinat titik potong garis AB dan garis CD adalah (-0.07,2.3)
\end{eulercomment}
\begin{eulercomment}

\end{eulercomment}
\eulersubheading{}
\begin{eulercomment}
2. Tentukan persamaan garis yang melalui titik potong x+2y=5 dan
2x-3y=-4 dengan gradien 3! Lalu Sketsakan!

Diketahui:\\
\end{eulercomment}
\begin{eulerformula}
\[
g := x+2y=5
\]
\end{eulerformula}
\begin{eulerformula}
\[
h := 2x-3y=-4
\]
\end{eulerformula}
\begin{eulerformula}
\[
m = 3
\]
\end{eulerformula}
\begin{eulercomment}
Ditanya:\\
Persamaan garis yang melalui titik potong kedua garis dengan m=3?

Jawab:\\
Akan ditentukan persamaan keluarga garis yang melalui titik potong dua
garis yang diberikan.\\
Persamaan garis :\\
\end{eulercomment}
\begin{eulerformula}
\[
Ax+By+C=0
\]
\end{eulerformula}
\begin{eulerformula}
\[
g := A1x+B1y+C1=0
\]
\end{eulerformula}
\begin{eulerformula}
\[
h := A2x+B2y+C2=0
\]
\end{eulerformula}
\begin{eulercomment}
Maka, akan terbentuk persamaan:\\
\end{eulercomment}
\begin{eulerformula}
\[
g + K.h = 0
\]
\end{eulerformula}
\begin{eulerformula}
\[
<=> (A1x+B1y+C1)+K(A2x+B2y+C2)= 0
\]
\end{eulerformula}
\begin{eulerformula}
\[
<=> (A1+K.A2)x + (B1+K.B2)y + C1 + K.C2 = 0
\]
\end{eulerformula}
\begin{eulercomment}
Akan ditentukan nilai K, melalui gradien:\\
\end{eulercomment}
\begin{eulerformula}
\[
g + K.h = 0
\]
\end{eulerformula}
\begin{eulerformula}
\[
(x+2y-5)+K(2x-3y+4)=0
\]
\end{eulerformula}
\begin{eulerformula}
\[
(1+2K)x+(2-3K)y +(-5+4K)=0
\]
\end{eulerformula}
\begin{eulercomment}
Persamaan garis dengan gradien 3:\\
\end{eulercomment}
\begin{eulerformula}
\[
m = 3
\]
\end{eulerformula}
\begin{eulerformula}
\[
- \frac {1+2K}{2-3K} = 3
\]
\end{eulerformula}
\begin{eulerformula}
\[
<=> 6-9K=-1-2K
\]
\end{eulerformula}
\begin{eulerformula}
\[
<=> 7K=7
\]
\end{eulerformula}
\begin{eulerformula}
\[
<=> K=1
\]
\end{eulerformula}
\begin{eulercomment}
Maka, diperoleh Persamaan garis yang melalui titik potong kedua garis
dengan gradien 3:\\
\end{eulercomment}
\begin{eulerformula}
\[
y=mx-K
\]
\end{eulerformula}
\begin{eulerformula}
\[
y=3x-1
\]
\end{eulerformula}
\begin{eulerformula}
\[
3x-y-1=0
\]
\end{eulerformula}
\begin{eulerprompt}
>setPlotRange(-5,5,-5,5);
\end{eulerprompt}
\begin{eulercomment}
Persamaan garis\\
\end{eulercomment}
\begin{eulerformula}
\[
x+2y=5
\]
\end{eulerformula}
\begin{eulercomment}
Titiknya :\\
\end{eulercomment}
\begin{eulerformula}
\[
x=0\ => y=\frac{5}{2}
\]
\end{eulerformula}
\begin{eulerformula}
\[
y=0\ => x=5
\]
\end{eulerformula}
\begin{eulerprompt}
>A=[0,2.5]; plotPoint(A);
>B=[5,0]; plotPoint(B);
>G=lineThrough(A, B); // garis melalui A dan B
>plotLine(G, "G");
\end{eulerprompt}
\begin{eulercomment}
Persamaan garis\\
\end{eulercomment}
\begin{eulerformula}
\[
2x-3y=-4
\]
\end{eulerformula}
\begin{eulercomment}
Titiknya :\\
\end{eulercomment}
\begin{eulerformula}
\[
x=0\ => y=\frac{4}{3}
\]
\end{eulerformula}
\begin{eulerformula}
\[
y=0\ => x=-2
\]
\end{eulerformula}
\begin{eulerprompt}
>C=[0,1.33]; plotPoint(C);
>D=[-2,0]; plotPoint(D);
>H=lineThrough(C, D); // garis melalui C dan D
>plotLine(H, "H");
\end{eulerprompt}
\begin{eulercomment}
Persamaan garis yang melalui titik potong garis G dan H dengan gradien
3\\
\end{eulercomment}
\begin{eulerformula}
\[
3x-y=1
\]
\end{eulerformula}
\begin{eulercomment}
Titiknya :\\
\end{eulercomment}
\begin{eulerformula}
\[
x=0\ => y=-1
\]
\end{eulerformula}
\begin{eulerformula}
\[
y=0\ => x=\frac{1}{3}
\]
\end{eulerformula}
\begin{eulerprompt}
>E=[0,-1]; plotPoint(E);
>F=[0.33,0]; plotPoint(F);
\end{eulerprompt}
\begin{eulercomment}
Misalkan persamaan garis hasilnya adalah P
\end{eulercomment}
\begin{eulerprompt}
>P=lineThrough(E, F); // garis melalui E dan F
>plotLine(P, "Garis P     ");
>M=lineIntersection(G,H);
>plotPoint(M):
\end{eulerprompt}
\eulerimg{27}{images/22305141017_Ardi Budi S_Geometri-207.png}
\begin{eulercomment}
\end{eulercomment}
\eulersubheading{}
\begin{eulercomment}
3. Lukislah lingkaran melalui ketiga titik berikut, kemudian tentukan
persamaannya.\\
(0,4); (0,-4); (6,0)

\end{eulercomment}
\begin{eulerprompt}
>setPlotRange(-7,7,-7,7);
>A = [0,4]; B = [0,-4]; C = [6,0]; // Mendefinisikan titik-titiknya
>&powerdisp:true;
>p1 &= x^2+A*x+y^2+B*y+C=0 with [x=0,y=4]
\end{eulerprompt}
\begin{euleroutput}
  
                             16 + 4 B + C = 0
  
\end{euleroutput}
\begin{eulerprompt}
>p2 &= x^2+A*x+y^2+B*y+C=0 with [x=0,y=-4]
\end{eulerprompt}
\begin{euleroutput}
  
                             16 - 4 B + C = 0
  
\end{euleroutput}
\begin{eulerprompt}
>p3 &= x^2+A*x+y^2+B*y+C=0 with [x=6,y=0]
\end{eulerprompt}
\begin{euleroutput}
  
                             36 + 6 A + C = 0
  
\end{euleroutput}
\begin{eulerprompt}
>$p2 + p1
\end{eulerprompt}
\begin{eulerformula}
\[
32+2\,C=0
\]
\end{eulerformula}
\begin{eulerprompt}
>$p1 - p2
\end{eulerprompt}
\begin{eulerformula}
\[
8\,B=0
\]
\end{eulerformula}
\begin{eulercomment}
Diperoleh C = 16 dan B = 0\\
Lalu, akan di subsitusikan ke salah satu persamaan p3
\end{eulercomment}
\begin{eulerprompt}
>nilaiA &= x^2+A*x+y^2+B*y+C=0 with [C=16,B=0,x=6,y=0]
\end{eulerprompt}
\begin{euleroutput}
  
                               52 + 6 A = 0
  
\end{euleroutput}
\begin{eulerprompt}
>&float(-26/3)
\end{eulerprompt}
\begin{euleroutput}
  
                           - 8.666666666666666
  
\end{euleroutput}
\begin{eulercomment}
Diperoleh A = 8,66\\
Sehingga, akan diperoleh persamaan lingkaran
\end{eulercomment}
\begin{eulerprompt}
>pl &= x^2+A*x+y^2+B*y+C=0 with [A=-8.66,B=0,C=16] 
\end{eulerprompt}
\begin{euleroutput}
  
                                       2    2
                        16 - 8.66 x + x  + y  = 0
  
\end{euleroutput}
\begin{eulercomment}
Maka, Persamaan lingkaran yang melalui titik (0,4), (0,-4), dan (6,0)
adalah:\\
\end{eulercomment}
\begin{eulerformula}
\[
16-8.66x+x^2+y^2=0
\]
\end{eulerformula}
\begin{eulerprompt}
>plotPoint(A);
>plotPoint(B);
>plotPoint(C);
>g = circleThrough(A,B,C);
>R = getCircleRadius(g);
>P = getCircleCenter(g)
\end{eulerprompt}
\begin{euleroutput}
  [1.66667,  0]
\end{euleroutput}
\begin{eulerprompt}
>plotPoint(P,"P")
>plotCircle(g,"16-8.66*x+x^2+y^2=0"):
\end{eulerprompt}
\eulerimg{27}{images/22305141017_Ardi Budi S_Geometri-211.png}
\begin{eulercomment}
\end{eulercomment}
\eulersubheading{}
\begin{eulercomment}
\end{eulercomment}
\end{eulernotebook}
\end{document}
